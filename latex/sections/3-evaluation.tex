\section{Evaluation}
\subsection{The Monte-Carlo method}
\label{sec:eval-circle}
The exact value of the unit circle's surface area is
\begin{equation}
    A_{\mathrm{circle,exact}}=\eval{\pi\,r^2}_{r=1}=\pi=3.141\dotso
    \label{eq:circle-area-exact}
\end{equation}
While results of single MC runs may deviate rather noticeably,
plotting the histogram of all MC runs reveals a normal distribution
centered around the exact value as expected from the central
limit theorem (see \cref{fig:circle-area-hist}).

\begin{figure}[ht]
    \centering
    \includegraphics[width=.75\textwidth]{circle-area-dist}
    \caption{Histogram of \num{1000} MC runs with mean and standard
    deviation \num{3.141(17)}}
    \label{fig:circle-area-hist}
\end{figure}

\subsection{Reweighting}
\label{sec:eval-rew}
One starts with computing the normalized histograms of the resulting spectra,
were the logarithm of the energies is used. The differential candidate number
$\dd{N}/\dd{E}$ can be approximated by dividing the number of particles for a
given bin by its width $\dd{E_i}=10^{\log(E_{i+1})-\log(E_{i})}$, where $E_{i},
E_{i+1}$ are the bin's edges; $\dd{N}/\dd{E}$ is plotted against the central
energies of the bins $\left(10^{\log(E_{i+1})}+10^{\log(E_{i})}\right)/2$ in
loglog-representation. The uncertainties are computed according to Poisson
statistics \cref{eq:mc-std}:
\begin{equation*}
    \Delta{y}=\frac{\dd{N}/\dd{E}}{\sqrt{N}}
\end{equation*}
were $N$ refers to the \emph{unweighted} and \emph{not normalized} number of
candidates per bin.

This is done for the energy spectra with $\alphaflat=1$ and
$\alphafermi=2$ and for the reweighted flat spectrum
($\alphaflat\to\alphafermi$, see \cref{eq:reweight}).

% \def\width{12.75cm}
% \def\height{9.56cm}
% \def\width{7.65cm}
% \def\height{5.74cm}
\def\width{6.8cm}
\def\height{5.1cm}
\def\binymin{3*10^-3}
\def\binymax{3}
\def\errymin{6*10^-25}
\def\errymax{6*10^-19}


\newcommand{\ReweightingAxis}[5]{%

\tikzexternalenable
\tikzsetnextfilename{ext-reweighting-#5}

\begin{tikzpicture}
    \pgfplotsset{set layers}

\begin{axis}[%
    width=\width, height=\height,
    scale only axis,
    axis y line*=left,
    axis x line=none,
    xmin={#3}, xmax={#4},
    ymin=\binymin, ymax=\binymax,
    ylabel={normalized number of events},
    ymode=log,
]
    \addplot [%
        ybar interval,
        ultra thin,
        fill=Paired-B,
        draw=Paired-B!20!white,
    ] table [x=bins,y=N]
        {reweighting-hist-alpha-#1-alpha_new-#2.csv};

\end{axis}

\begin{axis}[%
    width=\width, height=\height,
    scale only axis,
    axis y line*=right,
    xmin={#3}, xmax={#4},
    ymin=\errymin, ymax=\errymax,
    xlabel={$\log(E/\si{\eV})$},
    ylabel={$\dd{N}/\dd{E}/$a.u.},
    xtick distance=.5,
    grid=major,
    ymode=log,
    log basis y=10,
]
    \addplot [%
        draw=none,
        fill=none,
    ] table [%
        y={create col/linear regression={x=dE,y=dNdE}}
    ]
        {reweighting-errbar-alpha-#1-alpha_new-#2.csv};

    \addplot [%
        Paired-H,
        ultra thick,
        domain=18:22,
    ] {10^(\pgfplotstableregressiona*x+\pgfplotstableregressionb)};

    \addplot [%
        draw=none,
        error bars/.cd,
            y dir=both,y explicit,
            error bar style={ultra thick,color=black},
    ] table [x=dE,y=dNdE,y error=yerr]
        {reweighting-errbar-alpha-#1-alpha_new-#2.csv};

    \legend{,$\alpha_{\mathrm{fit}}=\pgfmathprintnumber{\pgfplotstableregressiona}$}
\end{axis}
\end{tikzpicture}

\tikzexternaldisable

}


% vim: set ff=unix tw=79 sw=4 ts=4 et ic ai :

\begin{figure}[t!]
    \centering
    \begin{subfigure}[t]{.5\textwidth}
        \centering
        \ReweightingAxis{-2.00}{None}{18.794}{21.54}{fermi}
        % \includegraphics[width=\textwidth]{extern/ext-reweighting-fermi}
        \caption{Fermi spectrum: direct simulation with $\alpha=2$}
        \label{fig:rew-fermi}
        \vspace*{\baselineskip}
    \end{subfigure}
% \end{figure}
% \begin{figure}[b!]
%     \ContinuedFloat
%     \centering
    \begin{subfigure}[t]{.5\textwidth}
        \centering
        \ReweightingAxis{-1.00}{None}{18.793}{21.795}{flat}
        % \includegraphics[width=\textwidth]{extern/ext-reweighting-flat}
        \caption{flat spectrum with $\alpha=1$}
        \label{fig:rew-flat}
        \vspace*{\baselineskip}
    \end{subfigure}
% \end{figure}
% \begin{figure}[t!]
%     \ContinuedFloat
%     \centering
    \begin{subfigure}[t]{.5\textwidth}
        \centering
        \ReweightingAxis{-1.00}{-2.00}{18.795}{21.793}{reweight}
        % \includegraphics[width=\textwidth]{extern/ext-reweighting-reweight}
        \caption{reweighted spectrum: simulated with $\alpha=1$, then
            reweighted to $\alpha=2$}
        \label{fig:rew-rew}
    \end{subfigure}

    \caption{Normalized histograms and differential energy spectra for
    different spectral indices}
    \label{fig:rew}
\end{figure}

The histogram of the Fermi spectrum (see \cref{fig:rew-fermi}) clearly shows
the predominance of particles at lower energies and an approx.~linear (in
the loglog-representation, slope $\sim-1$) decrease of particle numbers with
increasing energy. This is also reflected in the increasing uncertainties,
which scale as $\Delta{y}\propto1/\sqrt{N}$.
While the spectral curve behaves approx.~as expected from first-order Fermi
acceleration, a deviation is noticeable when looking at the fitted spectral
index.
Also note the \enquote{plateau} at the end of the spectrum, which is due to
insufficient MC statistics in that area.

On the other hand, the flat spectrum (see \cref{fig:rew-flat}) exhibits a
uniform distribution of particle numbers across all energies and therefor
consistently small uncertainties. When applying the weight
$w_{E_0}=E_0^{1-2}=E_0^{-1}$ to this distribution, one obtains a spectrum,
which is in good agreement with the theory, while also maintaining the
well-behaved uncertainties (\cref{fig:rew-rew}).

Due to the absence of any interactions, one would expect the observed spectrum
to be identical to the source spectrum, yet there are deviations in the fitted
spectral index. This is due to statistical effects from the MC simulations;
running these simulations multiple times would probably yield a normal
distribution around the \enquote{true} values. However these statistical
fluctuations are much more significant with the direct simulation (due to the
reasons discussed above), which additionally speaks in favor for the
reweighting.

\emph{Note:} Due to the clear advantages of this technique, all following
simulations will be performed with a flat spectrum and have their results
reweighted to correspond to Fermi-theory.


\subsection{Nuclear Interactions}
The influence of the individual interactions is analysed by looking at the
differential candidate number $\dd{N}/\dd{E}$ times $E^2$. Then, with a
log-scale on the x-axis, a power-law spectrum with $\alpha=2$ appears as a
horizontal line, which is convenient for identifying and discussing deviations
from the expected curve.
$\dd{N}/\dd{E}$ and the uncertainties are computed as described in
\cref{sec:eval-rew} and $E$ describes the central bin values. The results
are depicted in \cref{fig:interactions}.

\begin{figure}[ht!]
    \centering
    \newcommand{\PlotOneInteraction}[3]{%
    \edef\temp{%
        \noexpand\addplot+ [%
            thick,
            error bars/.cd,
                y dir=both,y explicit,
                error bar style={thick},
        ] table [x=dE,y=N,y error=yerr]
            {interactions-#1-#2-#3.csv};
    }
    \temp
}

\newcommand{\PlotAllInteractions}[2]{%
    \PlotOneInteraction{#1}{#2}{free}
    % \addlegendimage{empty legend}
    \pgfplotsforeachungrouped \kind in {%
            Redshift,
            PhotoPionCMB,
            PhotoPionIRB,
            ElectronPairCMB,
            ElectronPairIRB,
            PhotoDisCMB,
            PhotoDisIRB} {%
        \PlotOneInteraction{#1}{#2}{\kind}
    }
}

\tikzexternalenable
\tikzsetnextfilename{ext-interactions}

\begin{tikzpicture}
\begin{groupplot}[%
    group style={%
        % group name=energy loss plots,
        group size=2 by 2,
        xlabels at=edge bottom,
        ylabels at=edge left,
        xticklabels at=edge bottom,
    },
    xmode=log, ymode=linear,
    xlabel={$E/\si{\eV}$},
    ylabel={$E^2\dd{N}/\dd{E}/$a.u.},
    width=8.5cm,height=6.375cm,
    cycle list={[of colormap=Dark2]},
    grid=major,
    % title style={at={(.5,0)},below,yshift=-1em}
    title style={at={(0,1)},above,align=left,xshift=3em}
]
    \nextgroupplot[xmin=7e17,xmax=7e21,title={(a) H close~~}]
        \PlotAllInteractions{H}{close}

    \nextgroupplot[xmin=7e17,xmax=2e23,title={(b) \ce{^{56}Fe} close}]
        \PlotAllInteractions{Fe56}{close}

    \nextgroupplot[xmin=7e17,xmax=7e21,title={(c) H far~~~~}]
        \PlotAllInteractions{H}{far}

    \nextgroupplot[xmin=7e17,xmax=2e23,
            title={(d) \ce{^{56}Fe} far~~},
            legend to name={InteractionLegend},
            legend style={legend columns=2,
                /tikz/every even column/.append style={column sep=1.05cm}},
            legend cell align=left,
            legend transposed=true,
        ]
        \PlotAllInteractions{Fe56}{far}

    \legend{%
        free,Redshift,
        PhotoPionCMB,PhotoPionIRB,
        PhotoPairCMB,PhotoPairIRB,
        % ElectronPairCMB,ElectronPairIRB,
        PhotoDisCMB,PhotoDisIRB,
    }

\end{groupplot}

    \path (group c1r2.south east) --
        node[below,yshift=-1.2cm]{\ref*{InteractionLegend}}
        (group c2r2.south west);

\end{tikzpicture}

\tikzexternaldisable

% vim: set ff=unix tw=79 sw=4 ts=4 et ic ai :

    \caption{Normalized energy spectra $\dd{N}/\dd{E}\times{E^2}$ vs.~$E$ for
        all cases}
    \label{fig:interactions}
\end{figure}

When looking at the respective plots, there are a few features to recognize:
\begin{description}
    \item[H close]
        All curves are nearly horizontal for $E<\SI{3e20}{\electronvolt}$
        corresponding to an (unchanged) spectral index $\alpha\approx2$,
        \ie~no significant energy loss took place. Fluctuations are of
        statistical nature.

        At higher energies, \emph{Photo-Pion-Prod.~(CMB)} causes the
        spectrum first to flatten before steeply falling down at
        $E>\SI{1e21}{\electronvolt}$ (with $\alphafalling\approx2.31$). The CR
        flux is shifted to lower energies.

        The cutoff appears at higher energies than the estimated
        \SI{5e19}{\electronvolt}, which is no surprise here, because of the
        close source distances \SIrange{0}{10}{\mega\parsec}.

    \item[H far]
        The peak of shifted flux due to \emph{Photo-Pion-Prod.~(CMB)} is formed
        more sharply ($\alphafalling\approx5.58$), with no
        detected candidates beyond some maximum energy. However that maximum
        value lies with $\sim\SI{5e20}{\electronvolt}$ significantly further
        right than expected.

        The curve of \emph{Photo-Pair-Prod.~(CMB)} shows a weaker but
        noticeable decrease of flux without cutoff (with
        $\alphafalling\approx2.22$). While this behaviour is
        qualitatively expected, the starting point of this effect lies with
        $\sim\SI{2e19}{\electronvolt}$ also about one order of magnitude too
        far right.

    \item[Fe close]
        The cutoff due to \emph{Photo-Pion-Prod.~(CMB)} is also present (with
        $\alphafalling\approx2.96$) and
        lies about two orders of magnitude further right compared to hydrogen,
        which is a reasonable result, considering the higher number of
        nucleons on which the total energy is distributed. In addition to
        the peak of left-shifted flux immediately before the cutoff,
        another peak in the spectrum seems to emerge in the region
        \SIrange{1e20}{1e22}{\electronvolt}, which might be attributed to
        secondary particles resulting from \emph{Nuclear Decay} after the
        nucleus was converted to an unstable nuclide during
        Photo-Pion-Prod.~(\eg~${\text{p}+\gamma\to\text{n}+\pi^0}$ yielding
        \ce{^{56}Co} with a half-life of around 77 days; see also
        \cref{tab:secondaries}).

        This hypothesis could be supported by the curve of
        \emph{Photo-Disint.~(CMB)}, which exhibits a rather steep spectrum
        at $\sim\SI{5e21}{\electronvolt}$ resulting in a noticeable decrease
        of flux at energies above that value (with
        $\alphafalling\approx2.89$). Additionally a very prominent peak is
        apparent for lower energies. This flattening of the spectrum
        is probably caused by secondaries and products of nuclear
        decay with left-shifted energies.

    \item[Fe far]
        The aforementioned weak second peak in the curve of
        \emph{Photo-Pion-Prod.~(CMB)} is now present more distinctly
        ($\alphafalling\approx2.69$), while the first peak disappeared and the
        cutoff remains only in form of non-existent detections beyond some
        maximum energy, with the steep decline having completely disappeared.

        \emph{Photo-Pion-Prod.~(IRB)} also shows a peak at a similar
        location as with the CMB, followed by a slightly steeper spectrum,
        compared to the initial Fermi spectrum ($\alphafalling\approx2.12$).
        No cutoff is apparent.

        \emph{Photo-Disint.~(CMB)} now exhibits steep descent
        ($\alphafalling\approx3.40$) with a clear cutoff at
        $\sim\SI{2e21}{\electronvolt}$, meaning all \Fe-nuclei with larger
        energy have disintegrated or decayed into secondaries, causing the
        strong left-shifted peak in the spectrum.

        While \emph{Photo-Disint.~(IRB)} too shows a strong left-shifted
        peak reflecting the production of secondaries, no cutoff is present
        and the spectrum seems to return the initial spectral index after a
        steeper decline with $\alphafalling\approx2.45$.

        When given sufficient propagation time (\ie~longer distances),
        energy decay due to \emph{Photo-Pair-Prod.~(CMB)} becomes also
        noticeable for nuclei with energies greater than
        $\SI{1e21}{\electronvolt}$, apparent from a decline of the spectrum
        with $\alphafalling\approx3.63$. Since this process does not change
        the internal structure of the nuclei, no unstable products are produced
        and thus no decay happens. This is reflected by a missing peak of
        left-shifted flux in the spectrum.

\end{description}

In order to quantify the effects of the interaction mechanisms somewhat
more thoroughly and to compare them with each other, one can estimate a
\emph{loss time scale} by looking at the maximum energy loss and multiplying
the square root of the ratio of the corresponding mass and that energy
difference with some scale length:
\begin{equation}
    \tau\sim d_{\mathrm{scale}}\times\sqrt{\frac{m}{E_0-E_{\mathrm{final}}}}.
\end{equation}
This expresses the time during which the candidate experiences the given energy
loss (the shorter the time, the more significant the respective mechanism). In
order to obtain comparable results, for all distance regimes the same
scale length $d_{\mathrm{scale}}=\SI{100}{\mega\parsec}$ was chosen (see
\cref{tab:loss-time}).

\begin{table}[ht]
    \centering
    \sisetup{table-parse-only,table-column-width=10ex}
    \begin{tabular}{lSSSS}
        \toprule
        \multirow{2}{*}{Interaction}
        & \multicolumn{2}{c}{H} & \multicolumn{2}{c}{\Fe} \\
        \cmidrule{2-5}
        & {close} & {far} & {close} & {far} \\
        \midrule
        Redshift            & 2748  & 293   & 4000  & 428   \\
        Pion-Prod.~(CMB)    & 133   & 130   & 25    & 25    \\
        Pion-Prod.~(IRB)    & 192   & 135   & 25    & 25    \\
        Pair-Prod.~(CMB)    & 2494  & 221   & 861   & 185   \\
        Pair-Prod.~(IRB)    & 89419 & 8964  & 33313 & 2920  \\
        Photo-Disint.~(CMB) & {inf} & {inf} & 25    & 25    \\
        Photo-Disint.~(IRB) & {inf} & {inf} & 29    & 26    \\
        \bottomrule
    \end{tabular}
    \caption{Energy loss time scales for all cases in years}
    \label{tab:loss-time}
\end{table}

\begin{description}
    \item[Redshift]
        Energy loss due to the cosmological redshift of the source of course
        depends on the source's distance: for both candidate types the loss
        time of the far regime is one order of magnitude smaller than that of
        the close regime, which is reasonable.

        Further, this mechanism is
        independent of the particle's energy, and thus does not change the
        spectral index.

    \item[Pion-Production]
        As a proton with sufficient energy is estimated to loss all of its energy
        after $\sim\SI{30}{\mega\parsec}$, which is of the same order of
        magnitude as the close distance regime, the similar loss times for both
        distance regimes are reasonable.

        Since $\tau\propto\sqrt{m}$, one might expect longer iron time scales,
        however the opposite is the case. This could probably be attributed
        to the higher number of nucleons, which can undergo this process and
        additional nuclear decay of unstable secondaries (see also
        \cref{tab:secondaries}).

        While the IRB interaction is comparable to the CMB interaction in all
        cases, it is more significant for hydrogen on larger distances. This
        might be due to a lower photon density and a therewith smaller cross
        section. This is again compensated by iron's higher nucleon number.

    \item[Pair-Production]
        As expected from the smaller fractional energy loss and as already seen
        in \cref{fig:interactions}, pair production makes itself noticeable not
        until larger distances.

        While comparable on the large distance regime, the shorter time scale
        for iron on the close regime might again be explained by its higher cross
        section.

        While for both candidate types the time scales in interaction with the
        IRB are rather large (and thus less significant), their difference is
        probably also due to the lower photon density and iron's higher cross
        section.

    \item[Photo-Disintegration]
        The infinite time scales for hydrogen show that no energy loss
        occurred, since there is no nucleus which can disintegrate into smaller
        nuclei (as expected from theory; this simulation could have been
        omitted).

        The values for iron are comparable to energy loss due to
        pion-production and thus belong to the more significant processes (as
        already seen in \cref{fig:interactions}). Interaction with IRB takes
        slightly longer on close distances.
\end{description}

\begin{table}[p]
    \centering
    \newcolumntype{E}{>{\zzz}l<{\relax}@{}}
    \def\zzz#1\relax{\ce{#1}}
    \begin{tabular}{lESES}
        \toprule
        \multirow{2}{*}{Interaction}
        & \multicolumn{2}{c}{close} & \multicolumn{2}{c}{far} \\
        \cmidrule{2-5}
        & {Product} & {~Abundance/\%} & {Product} & {~Abundance/\%} \\
        \midrule
%
        \multirow{3}{*}{\shortstack[l]{%
            Pion-Prod.~(CMB)\\
            \footnotesize
            close: 16.43\% decayed \\
            \footnotesize
            far: 23.03\% decayed
        }}
        & ^1H & 79.38       & ^1H & 99.37 \\
        & {n} & 15.38         & {traces} & 0.63 \\
        & {traces} & 5.24   & & \\
        \cmidrule{1-5}
%
        \multirow{7}{*}{\shortstack[l]{%
            Pion-Prod.~(IRB) \\
            \footnotesize
            close: 3.03\% decayed \\
            \footnotesize
            far: 42.16\% decayed
        }}
        & ^1H & 48.49       & ^1H & 90.86 \\
        & ^{55}Fe^* & 23.05   & {traces} & 9.14 \\
        & ^{55}Mn & 21.46   & & \\
        & {n} & 3.34          & & \\
        & ^{54}Mn^* & 2.23    & & \\
        & ^{54}Fe & 1.11    & & \\
        & {traces} & 0.32   & & \\
        \cmidrule{1-5}
%
        \multirow{5}{*}{\shortstack[l]{%
            Photo-Disint.~(CMB) \\
            \footnotesize
            close: 39.35\% decayed \\
            \footnotesize
            far: 44.86\% decayed
        }}
        & ^1H & 87.73       & ^1H & 99.01 \\
        & {n} & 5.88          & {traces} & 0.99 \\
        & ^4He & 2.12       & & \\
        & ^3He & 1.89       & & \\
        & {traces} & 2.38   & & \\
        \cmidrule{1-5}
%
        \multirow{9}{*}{\shortstack[l]{%
            Photo-Disint.~(IRB) \\
            \footnotesize
            close: 6.28\% decayed \\
            \footnotesize
            far: 59.02\% decayed
        }}
        & ^1H & 62.67       & ^1H & 89.17 \\
        & ^{55}Fe^* & 16.36   & ^4He & 3.02 \\
        & ^{54}Fe & 4.89    & ^3He & 1.82 \\
        & ^{53}Mn^* & 3.58    & {traces} & 6.00 \\
        & ^{55}Mn & 3.52    & & \\
        & ^{54}Mn^* & 3.07    & & \\
        & ^{52}Cr & 1.76    & & \\
        & {n} & 1.02        & & \\
        & {traces} & 3.13   & & \\
        \bottomrule
    \end{tabular}
    \caption[Secondaries of \Fe]{%
    \textbf{Secondaries of \Fe:}
    The decay of an iron nucleus may be triggered by Photo-Pion-Production and
    Photo-Disintegration.

    For each simulation \num{10000} candidates were propagated, of which the
    specified percentage decayed. The isotope designations of the decay
    products were determined from the recorded candidate IDs and their
    composition is listed here. Radioactive isotopes are indicated by an
    asterisk.

    Note the absence of neutrons in the far regime: this is due to free
    neutrons being unstable with a half life of \SI{881.5(15)}{\second}.}
    \label{tab:secondaries}
\end{table}



\subsection{Magnetic Deflection}
\label{sec:eval-defl}
The deflections experienced by the trajectories of CRs due to turbulent
magnetic fields are investigated by discussing the
\begin{enumerate}[label=\textbf{(\roman*)}]
    \item the differential candidate number per rigidity $\dd{N}/\dd{\rho}$
        times $\rho^2$, with $\rho=\Lamor/l_c$, $\Lamor=E/ZeBc$ and its
        deviation from the initial power-law spectrum with $\alpha=2$
        (the spectrum is obtained as described in \cref{sec:eval-rew}; see
        \cref{fig:deflection-spectra}),
    \item the distributions of trajectory lengths (see
        \cref{fig:deflection-traj}), and
    \item the distributions of deflection angles between the momentum vectors
        initially at the source and at the point of detection (see
        \cref{fig:deflection-defl}).
\end{enumerate}

But at first, regarding some important numbers:
\begin{itemize}
    \item With $\lmin=\SI{150}{\kilo\parsec}$ and
        $\lmax=\SI{2000}{\kilo\parsec}$ the expected resonant regime is
        \[\lmin/\lmax=0.075<\rho<1\]

    \item Further, with $\lmin$ and $\lmax$ given, \CRPropa~calculates the
        correlation length to $l_c=\SI{.48}{\mega\parsec}$, from which one has
        the radii of the six observer spheres:
        \[\Robserver=\SIlist{24e-3;0.24;0.48;2.4;24;48}{\mega\parsec}\]
        These distances are the minimally possible trajectory lengths,
        corresponding to candidates travelling without deflection from the
        source to the point of detection.

        In the plots, this is represented by the first bins with
        \[\log(\Robserver/\si{\mega\parsec})=
            \numlist{-1.62;-0.62;-0.32;0.38;1.38;1.68}.\]

    \item The physical upper limit for the trajectory lengths is
        $\dAoU=\SI{4231}{\mega\parsec}$, which is the light-travel
        distance corresponding to the age of the universe with
        \SI{13.8e9}{\jahr}. This is enforced by setting a maximum trajectory
        length as a termination condition for the simulation.
        However, due to an index error in the steering script, not the second
        largest, but the largest radius was chosen for its computation,
        resulting in $\dmax=\SI{4800}{\mega\parsec}$, corresponding to a
        light-travel distance of \SI{15.7e9}{\jahr}.

        In the plots, these values are represented by the last bins with
        \[\log(\dAoU/\si{\mega\parsec})=3.63
            \qq{and}\log(\dmax/\si{\mega\parsec})=3.68.\]
\end{itemize}

\begin{description}
    \item[Rigidity Spectra]
        The resulting spectra are plotted in \cref{fig:deflection-spectra}.
        Resonant scattering shows itself as peaks in those spectra,
        \ie~the flux at resonant rigidities is amplified (candidates below the
        peak gain, while those above lose energy), which results in a deviation
        from $\alpha=2$.

        With increasing radii the resonant regime is shifted to higher
        rigidities.
        This regime seems to be identical for all field strengths at a fixed
        radius. Further, the estimated spectral indices of the falling edges of
        the resonant peaks at a given radius are very similar (see
        \cref{tab:defl}; the exception being $\Brms=\SI{1}{\nano\gauss}$ and
        $\SI{10}{\nano\gauss}$ at $\Robserver=\SI{24e-3}{\mega\parsec}$, which
        is due to those curves not yet falling into the resonant interval).

        Since $\rho\propto1/\Brms$, the curves for smaller field strengths are
        relatively shifted to higher rigidities according to their respective
        scale factor.
        Because of this shift the curves of smaller field strengths fall into
        the resonant regime later, thus exhibiting less scattering in the
        investigated interval, while having the same energy.

    \item[Trajectory Lengths]
        The distributions of trajectory lengths observed at the six observer
        spheres are plotted in \cref{fig:deflection-traj}.
        Their physical limitations and how they are represented in the plots is
        discussed above.

        For small radii, the first bins (\ie~undeflected candidates) dominate
        the distributions. For $\Brms\ge\SI{10}{\nano\gauss}$ a small irregular
        peak is found at longer trajectories (the irregularities being due to
        statistics, because only a few particles are scattered at this point).
        With increasing radii, this peak broadens, grows more regular (as more
        particles are scattered) and its maximum shifts to the right. The
        initial peak at the first bin decreases in height and the local minimum
        between the two peaks becomes more distinct.

        This development continues with the second peak shifting eventually out
        of the range of possible trajectories and particles accumulating at the
        high end of the spectrum, almost resembling a power law.
        At the farthest radius the initial peak is still recognizable for all
        field strengths, but at different intensities: for
        $\Brms=\SI{50}{\nano\gauss}$ it almost vanishes, while for
        $\Brms=\SI{1}{\nano\gauss}$ it is still stronger than the shifted peak.
        Additionally, while the rising edge of the shifted peak is rather
        regular for $\Brms\ge\SI{10}{\nano\gauss}$, for
        $\Brms=\SI{10}{\nano\gauss}$ statistical irregularities remain
        recognizable.

        \emph{Note:} Since candidates are not removed upon detection, they can
        be detected multiple times by the same sphere and contribute with
        larger trajectories in the resonant case.

    \item[Deflection Angles]
        The distributions of deflection angles observed at the six observer
        spheres are plotted in \cref{fig:deflection-defl}. A range from
        $\ablenkung=\SI{0}{\degree}$ (no deflection) to
        $\ablenkung=\SI{180}{\degree}$ (candidate completely inverses direction
        of travel) is covered.

        The development for increasing radii is similar as above: at first one
        has a strong peak in the first bins (\ie~minimal deflection) with a
        smaller irregular peak for $\Brms\ge\SI{10}{\nano\gauss}$. This peaks
        shifts to the right, grows broader and more regular, eventually forming
        a half-circle (in semilog-representation) centered around
        \SI{90}{\degree}.
        For larger field strengths the initial peak sooner or later disappears,
        while it remains noticeable for $\Brms=\SI{1}{\nano\gauss}$.
\end{description}

When comparing these observations, there are a few things to note:
\begin{itemize}
    \item The development of the angular and trajectory length distributions
        for increasing radii is reflected by the averages those distributions
        (listed in \cref{tab:defl}).

        The averages of the trajectory lengths quickly grow (approx.~like a
        power law), while those for different field strengths at a fixed radius
        lie within the same order of magnitude (the exception being
        $\Brms=\SI{1}{\nano\gauss}$ at some instances).

        For the deflection angles the averages converge to \SI{90}{\degree},
        with $\Brms=\SI{1}{\nano\gauss}$ lagging behind due to the remaining
        initial peak.

    \item The large standard deviations of the distributions reflect on their
        spread-out character with sometimes multiple peaks. For the trajectory
        lengths at $\Robserver<\SI{24}{\mega\parsec}$ and for the deflection
        angles at $\Robserver<\SI{2.4}{\mega\parsec}$ the standard deviations
        are distinctly larger than the averages (in the first case even by an
        order of magnitude!) and therefore are not suitable to construct a
        confidence interval in order to describe the distribution: this would
        result in non-physical results (\ie~$\traj<0$).

    \item Some resemblance between the developments of all three quantities
        (rigidity spectra, trajectory lengths and deflection angles) can be
        identified:

        At small radii, the scattering due to $\Brms=\SI{50}{\nano\gauss}$
        clearly dominates, but the difference with the other cases
        $\Brms\ge\SI{10}{\nano\gauss}$ quickly becomes less significant as they
        grow highly similar to each other (except for the initial peak in the
        trajectory length distribution and a small segment at the higher end in
        the rigidity spectrum).

        The interesting case here is $\Brms=\SI{1}{\nano\gauss}$, whose
        spectrum remains undisturbed for $\rho>\num{5e1}$
        (\ie~$E>\SI{2.2e19}{\eV}$). This un-scattered flux causes the peaks of
        the initial bins in the other two distributions. These parameters
        could provide an approach to explain the observed anisotropy in the
        arrival directions at those high energies.

\end{itemize}

\begin{table}[ht!]
    \centering
    \begin{tabular}{%
        SSS
        S[table-format=4.2,table-figures-uncertainty=1,table-number-alignment=center]
        S[table-format=6.2,table-figures-uncertainty=1,table-number-alignment=center]
    }
        \toprule
        {\multirow{2}{*}{\Robserver/\si{\mega\parsec}}} &
        {\multirow{2}{*}{\Brms/\si{\nano\gauss}}} &
        \multicolumn{3}{c}{Measurements} \\
        \cmidrule{3-5}
        & & {\alphafalling} & {\traj/\si{\mega\parsec}} & {\ablenkung/\si{\degree}}
        \\
        \midrule
        {\multirow{4}{*}{\num{24e-3}}}
        & 1     & 2.00 & 0.03(26)      & 0.49(161)     \\
        & 10    & 2.04 & 0.17(619)     & 08.16(1845)   \\
        & 20    & 3.07 & 0.23(420)     & 20.46(3242)   \\
        & 50    & 3.08 & 00.88(3366)   & 48.93(4602)   \\
        \cmidrule{1-5}
%
        {\multirow{4}{*}{\num{.24}}}
        & 1     & 2.22 & 01.70(5405)   & 10.24(2037)   \\
        & 10    & 2.31 & 019.75(16619) & 49.20(4726)   \\
        & 20    & 2.29 & 026.88(21741) & 59.24(4756)   \\
        & 50    & 2.35 & 040.78(21228) & 73.38(4535)   \\
        \cmidrule{1-5}
%
        {\multirow{4}{*}{\num{.48}}}
        & 1     & 2.47 & 006.89(10490) & 20.44(3270)   \\
        & 10    & 2.38 & 055.17(29546) & 63.12(4727)   \\
        & 20    & 2.39 & 071.12(31868) & 72.39(4562)   \\
        & 50    & 2.40 & 128.14(46309) & 82.15(4249)   \\
        \cmidrule{1-5}
%
        {\multirow{4}{*}{\num{2.4}}}
        & 1     & 3.03 & 106.87(41466) & 50.54(4807)   \\
        & 10    & 2.99 & 362.74(71616) & 83.49(4188)   \\
        & 20    & 3.00 & 419.58(77142) & 86.59(4094)   \\
        & 50    & 2.97 & 526.08(86515) & 88.17(3956)   \\
        \cmidrule{1-5}
%
        {\multirow{4}{*}{\num{24}}}
        & 1     & 3.30 & 1361.81(125738) & 80.35(4461) \\
        & 10    & 3.35 & 2378.69(130089) & 88.25(3975) \\
        & 20    & 3.35 & 2527.83(127112) & 89.70(3929) \\
        & 50    & 3.29 & 2669.64(127207) & 89.65(3924) \\
        \cmidrule{1-5}
%
        {\multirow{4}{*}{\num{48}}}
        & 1     & 3.23 & 2006.20(145893) & 80.25(4411) \\
        & 10    & 3.18 & 2646.78(141007) & 86.68(3996) \\
        & 20    & 3.22 & 2709.92(140225) & 88.53(3928) \\
        & 50    & 3.37 & 2726.14(135104) & 89.52(3906) \\
        \bottomrule
    \end{tabular}
    \caption{%
        Spectral index of the falling edge of the resonance peak \alphafalling,
        averaged trajectory length \traj~and averaged deflection angle
        \ablenkung~for each observer sphere (given by \Robserver) and for each
        simulation (given by \Brms)
    }
    \label{tab:defl}
\end{table}

\begin{figure}[p]
    \centering
    \newcommand{\PlotOneDeflectionSpectrum}[2]{%
    \edef\temp{%
        \noexpand\addplot+ [%
            thick,
            error bars/.cd,
                y dir=both,y explicit,
                error bar style={thick},
        ] table [x=dE,y=N,y error=yerr]
            {deflections_error_brms-#2_r-#1.csv};
    }
    \temp
}

\newcommand{\PlotOneSpectrum}[1]{%
    \pgfplotsforeachungrouped \bfield in {%
            1e-13,
            1e-12,
            2e-12,
            5e-12} {%
        \PlotOneDeflectionSpectrum{#1}{\bfield}
    }
}


\tikzexternalenable
\tikzsetnextfilename{ext-defl-spectrum}

\begin{tikzpicture}
\begin{groupplot}[%
    group style={%
        % group name=energy loss plots,
        group size=2 by 3,
        xlabels at=edge bottom,
        ylabels at=edge left,
        xticklabels at=edge bottom,
        % yticklabels at=edge left,
    },
    xmin=2e-2,xmax=2e3,
    xmode=log, ymode=log,
    xlabel={$\rho$},
    ylabel={$\rho^2\times\dd{N}/\dd{\rho}/$a.u.},
    width=8.5cm,height=6.375cm,
    cycle list={[of colormap=Dark2]},
    grid=major,
    % title style={at={(.5,0)},below,yshift=-1em}
    title style={above,align=center}
]
    \nextgroupplot[%
        % xmin=3e-1,xmax=2e3,
        ymin=1e-2,ymax=5e-1,
        title={(a)~$\Robserver=\SI{24e-3}{\mega\parsec}$}]
    \PlotOneSpectrum{7e+20}

    \nextgroupplot[%
        % xmin=3e-1,xmax=2e3,
        ymin=1e-2,ymax=5e-1,
        title={(b)~$\Robserver=\SI{.24}{\mega\parsec}$}]
    \PlotOneSpectrum{7e+21}

    \nextgroupplot[%
        % xmin=3e-1,xmax=2e3,
        ymin=7e-3,ymax=5e-1,
        title={(c)~$\Robserver=\SI{.48}{\mega\parsec}$}]
    \PlotOneSpectrum{1e+22}

    \nextgroupplot[%
        % xmin=3e-1,xmax=2e3,
        ymin=2e-3,ymax=7e-1,
        title={(d)~$\Robserver=\SI{2.4}{\mega\parsec}$}]
    \PlotOneSpectrum{7e+22}

    \nextgroupplot[%
        % xmin=3e-1,xmax=2e3,
        ymin=3e-3,ymax=6e-1,
        title={(e)~$\Robserver=\SI{24}{\mega\parsec}$}]
    \PlotOneSpectrum{7e+23}

    \nextgroupplot[%
        % xmin=3e-1,xmax=2e3,
        ymin=1e-2,ymax=2e0,
        title={(f)~$\Robserver=\SI{48}{\mega\parsec}$},
        legend to name={DeflectionSpectraLegend},
        legend style={legend columns=2,
            /tikz/every even column/.append style={column sep=2em}},
        legend cell align=left,
        legend transposed=true,
    ]
    \PlotOneSpectrum{1e+24}

    \legend{%
        $(\Brms=\SI{1}{\nano\gauss})\times10^{-1}$,
        $\Brms=\SI{10}{\nano\gauss}$,
        $\Brms=\SI{20}{\nano\gauss}$,
        $\Brms=\SI{50}{\nano\gauss}$,
    }

\end{groupplot}

    \path (group c1r3.south east) --
        node[below,yshift=-3em]{\ref*{DeflectionSpectraLegend}}
        (group c2r3.south west);

\end{tikzpicture}

\tikzexternaldisable

% vim: set ff=unix tw=79 sw=4 ts=4 et ic ai :

    \caption[rigidity spectra]{%
        Normalized energy spectra (in terms of the rigidity $\rho=\Lamor/l_c$)
        $\rho^2\times\dd{N}/\dd{\rho}$ vs.~$\rho$.

        \emph{Note:} The curve of $\Brms=\SI{1}{\nano\gauss}$ has been
        multiplied by a scale factor $10^{-1}$ to move it closer to the other
        curves for better comparability.
    }
    \label{fig:deflection-spectra}
\end{figure}

\begin{figure}[p]
    \centering
    \newcommand{\PlotOneDeflectionTrajectory}[2]{%
    \edef\temp{%
        \noexpand\addplot+ [%
            const plot mark mid,
            thick,
        ] table [x=bins,y=N]
            {deflections_traj_brms-#2_r-#1.csv};
    }
    \temp
}

\newcommand{\PlotOneSphereTraj}[1]{%
    \pgfplotsforeachungrouped \bfield in {%
            1e-13,
            1e-12,
            2e-12,
            5e-12} {%
        \PlotOneDeflectionTrajectory{#1}{\bfield}
    }
}


\tikzexternalenable
\tikzsetnextfilename{ext-defl-traj}

\begin{tikzpicture}
\begin{groupplot}[%
    group style={%
        % group name=energy loss plots,
        group size=2 by 3,
        xlabels at=edge bottom,
        ylabels at=edge left,
        % xticklabels at=edge bottom,
        yticklabels at=edge left,
        horizontal sep=2em,
        vertical sep=4em,
    },
    % xmin=2e-2,xmax=2e3,
    xmode=linear, ymode=log,
    xlabel={$\log(\traj/\si{\mega\parsec})$},
    ylabel={relative abundance},
    width=8.5cm,height=6.375cm,
    cycle list={[of colormap=Dark2]},
    grid=major,
    % title style={at={(.5,0)},below,yshift=-1em}
    title style={above,align=center}
]
    \nextgroupplot[%
        ymin=4e-4,ymax=2e1,
        title={(a)~$\Robserver=\SI{24e-3}{\mega\parsec}$}]
    \PlotOneSphereTraj{7e+20}

    \nextgroupplot[%
        % ymin=4e-4,ymax=1.5e1,
        ymin=4e-4,ymax=2e1,
        title={(b)~$\Robserver=\SI{.24}{\mega\parsec}$}]
    \PlotOneSphereTraj{7e+21}

    \nextgroupplot[%
        ymin=1e-3,ymax=1.5e1,
        title={(c)~$\Robserver=\SI{.48}{\mega\parsec}$}]
    \PlotOneSphereTraj{1e+22}

    \nextgroupplot[%
        % ymin=2e-2,ymax=1e1,
        ymin=1e-3,ymax=1.5e1,
        title={(d)~$\Robserver=\SI{2.4}{\mega\parsec}$}]
    \PlotOneSphereTraj{7e+22}

    \nextgroupplot[%
        % ymin=1e-2,ymax=3e0,
        ymin=1e-2,ymax=4e0,
        title={(e)~$\Robserver=\SI{24}{\mega\parsec}$}]
    \PlotOneSphereTraj{7e+23}

    \nextgroupplot[%
        % ymin=2e-2,ymax=4e0,
        ymin=1e-2,ymax=4e0,
        title={(f)~$\Robserver=\SI{48}{\mega\parsec}$},
        legend to name={DeflectionTrajLegend},
        legend style={legend columns=2,
            /tikz/every even column/.append style={column sep=2em}},
        legend cell align=left,
        legend transposed=true,
    ]
    \PlotOneSphereTraj{1e+24}

    \legend{%
        $\Brms=\SI{1}{\nano\gauss}$,
        $\Brms=\SI{10}{\nano\gauss}$,
        $\Brms=\SI{20}{\nano\gauss}$,
        $\Brms=\SI{50}{\nano\gauss}$,
    }

\end{groupplot}

    \path (group c1r3.south east) --
        node[below,yshift=-3em]{\ref*{DeflectionTrajLegend}}
        (group c2r3.south west);

\end{tikzpicture}

\tikzexternaldisable

% vim: set ff=unix tw=79 sw=4 ts=4 et ic ai :

    \caption{Normalized distribution of the trajectory lengths \traj~in
    loglog-representation}
    \label{fig:deflection-traj}
\end{figure}

\begin{figure}[p]
    \centering
    \newcommand{\PlotOneDeflection}[2]{%
    \edef\temp{%
        \noexpand\addplot+ [%
            const plot mark mid,
            thick,
        ] table [x=bins,y=N]
            {deflections_defl_brms-#2_r-#1.csv};
    }
    \temp
}

\newcommand{\PlotOneSphere}[1]{%
    \pgfplotsforeachungrouped \bfield in {%
            1e-13,
            1e-12,
            2e-12,
            5e-12} {%
        \PlotOneDeflection{#1}{\bfield}
    }
}


\tikzexternalenable
\tikzsetnextfilename{ext-defl-defl}

\begin{tikzpicture}
\begin{groupplot}[%
    group style={%
        % group name=energy loss plots,
        group size=2 by 3,
        xlabels at=edge bottom,
        ylabels at=edge left,
        % xticklabels at=edge bottom,
        yticklabels at=edge left,
        horizontal sep=2em,
        vertical sep=4em,
    },
    % xmin=2e-2,xmax=2e3,
    xmode=linear, ymode=log,
    xlabel={$\ablenkung/\si{\degree}$},
    ylabel={relative abundance},
    width=8.5cm,height=6.375cm,
    cycle list={[of colormap=Dark2]},
    grid=major,
    % title style={at={(.5,0)},below,yshift=-1em}
    title style={above,align=center}
]
    \nextgroupplot[%
        ymin=2e-5,ymax=7e-1,
        title={(a)~$\Robserver=\SI{24e-3}{\mega\parsec}$}]
    \PlotOneSphere{7e+20}

    \nextgroupplot[%
        ymin=2e-5,ymax=7e-1,
        title={(b)~$\Robserver=\SI{.24}{\mega\parsec}$}]
    \PlotOneSphere{7e+21}

    \nextgroupplot[%
        ymin=10e-5,ymax=2e-1,
        title={(c)~$\Robserver=\SI{.48}{\mega\parsec}$}]
    \PlotOneSphere{1e+22}

    \nextgroupplot[%
        ymin=10e-5,ymax=2e-1,
        title={(d)~$\Robserver=\SI{2.4}{\mega\parsec}$}]
    \PlotOneSphere{7e+22}

    \nextgroupplot[%
        ymin=2e-4,ymax=1.5e-2,
        title={(e)~$\Robserver=\SI{24}{\mega\parsec}$}]
    \PlotOneSphere{7e+23}

    \nextgroupplot[%
        ymin=2e-4,ymax=1.5e-2,
        title={(f)~$\Robserver=\SI{48}{\mega\parsec}$},
        legend to name={DeflectionDeflLegend},
        legend style={legend columns=2,
            /tikz/every even column/.append style={column sep=2em}},
        legend cell align=left,
        legend transposed=true,
    ]
    \PlotOneSphere{1e+24}

    \legend{%
        $\Brms=\SI{1}{\nano\gauss}$,
        $\Brms=\SI{10}{\nano\gauss}$,
        $\Brms=\SI{20}{\nano\gauss}$,
        $\Brms=\SI{50}{\nano\gauss}$,
    }

\end{groupplot}

    \path (group c1r3.south east) --
        node[below,yshift=-3em]{\ref*{DeflectionDeflLegend}}
        (group c2r3.south west);

\end{tikzpicture}

\tikzexternaldisable

% vim: set ff=unix tw=79 sw=4 ts=4 et ic ai :

    \caption{Normalized distribution of the deflection angles \ablenkung}
    \label{fig:deflection-defl}
\end{figure}



\subsection{Constraints on the Sources}
\subsubsection*{Prerequisites}
Modeling the simulation setup naively with one or more sources and a
small observer sphere representing the earth results in a considerable
computational effort in order to obtain significant statistics,
since only very few candidates would actually reach that observer. The setup
used here is known as \emph{inverted setup:} One point-like source in the
center of the domain is surrounded by a large observer sphere, which detects
all propagated candidates (\emph{original setup}). The results are then
transformed into the results of a proper 3D simulation, with a single source
outside of a single small observer, such as the earth (\emph{proper setup}).
The vector between source and point of detection (\emph{detection vector}) in
the original setup corresponds to the proper line of sight in the proper setup,
its length (\ie~the radius of the large observer sphere) to the proper source
distance and its angle with the final momentum vector to the proper deflection
angle \ablenkung~of the detected candidate (see the sky maps in
\cref{fig:skymaps} for a visualization).


% \def\width{8.5cm}
% \def\width{7.65cm}
% \def\height{5.75cm}

\newcommand{\SkyMapHist}[3]{%

\def\width{#3}
\def\height{.75\width}

\tikzexternalenable
\tikzsetnextfilename{ext-skymap-#1}

\begin{tikzpicture}
\begin{axis}[%
    width=\width, height=\height,
    xlabel={$\ablenkung/\si{\degree}$},
    ylabel={relative abundance},
    ymode=log,
    grid=major,
    % cycle list={[of colormap=Dark2]},
    ymin=#2,
]
    \addplot [%
        const plot mark mid,
        thick,
        no markers,
        draw=Dark2-A,
    ] table [x=bins, y=N] {%
        plotdata/sky_map_hist_brms-1e-12_r-#1.csv
    };
\end{axis}
\end{tikzpicture}

\tikzexternaldisable
}

% vim: set ff=unix tw=79 sw=4 ts=4 et ic ai :

\begin{figure}[p]
    \centering
    \begin{subfigure}[t]{\textwidth}
        \centering
        \caption{$\Robserver=\SI{.24}{\mega\parsec}$}
        \label{fig:skymap-close}
        \begin{minipage}{.48\textwidth}
            \includegraphics[width=\textwidth]{sky_map_brms-1e-12_r-7e+21}
        \end{minipage}~%
        \begin{minipage}{.48\textwidth}
            \SkyMapHist{7e+21}{5e0}{\textwidth}
        \end{minipage}
    \end{subfigure}
%
    \begin{subfigure}[t]{\textwidth}
        \centering
        \caption{$\Robserver=\SI{2.4}{\mega\parsec}$}
        \label{fig:skymap-medium}
        \begin{minipage}{.48\textwidth}
            \includegraphics[width=\textwidth]{sky_map_brms-1e-12_r-7e+22}
        \end{minipage}~%
        \begin{minipage}{.48\textwidth}
            \SkyMapHist{7e+22}{4e1}{\textwidth}
        \end{minipage}
    \end{subfigure}
%
    \begin{subfigure}[t]{\textwidth}
        \centering
        \caption{$\Robserver=\SI{24}{\mega\parsec}$}
        \label{fig:skymap-far}
        \begin{minipage}{.48\textwidth}
            \includegraphics[width=\textwidth]{sky_map_brms-1e-12_r-7e+23}
        \end{minipage}~%
        \begin{minipage}{.48\textwidth}
            \SkyMapHist{7e+23}{1e2}{\textwidth}
        \end{minipage}
    \end{subfigure}
    \caption[Sky Maps]{%
        Sky maps of a proper 3D simulation with one source as seen from the
        observer: For each candidate the coordinate system was rotated such
        that the detection vector lies parallel to the x-axis
        ($\phi_r=0,\theta_r=0$ in spherical coordinates). The polar and
        azimutal angles $\phi_p$ and $\theta_p$ of the corresponding momentum
        vectors were then accumulated in a histogram and plotted in a
        Hammer-Aitoff projection.

        For three observer distances
        $\Robserver=\SIlist{.24;2.4;24}{\mega\parsec}$ with
        $\Brms=\SI{10}{\nano\gauss}$ the resulting sky maps and deflection
        distributions were plotted to illustrate different degrees of
        anisotropy.
    }
    \label{fig:skymaps}
\end{figure}


The quantity of interest here is the particle flux
\begin{equation*}
    \CRFlux=\frac{\dd{N}}{\dd{A}\dd{\Omega}\dd{t}}.
\end{equation*}
To obtain the differential particle number $\dd{N}$ the histogram of the
energies is computed, with additional weights $1/\abs{\cos\ablenkung}$ to account
for the angular dependence of the effective area $\dd{A}$ and $1/\sin\ablenkung$ to
account for the solid angle $\dd{\Omega}\propto\sin\ablenkung$.
Further, the effective area in spherical coordinates is known to be
$\dd{A}=r^2\dd{\Omega}$, thus the particle flux becomes
\begin{equation*}
    \CRFlux=\frac{\Robserver^2\dd{N}}{\left(\dd{A}\right)^2},
\end{equation*}
where \Robserver~is the proper source distance and
$\dd{A}=4\pi{r}^2$ is the surface area of the proper small observer
sphere, with $r=\SI{25}{\kilo\parsec}$ being chosen as the
estimated radius of the Milky Way. Finally, $\dd{t}=1$ was set, which does not
influence the results, as the histogram is normalized anyway.
This expression can then straightforwardly be computed, and the corresponding
uncertainties $\Delta{N}$ are found as described in \cref{sec:eval-rew}.

\subsubsection*{Evaluation Procedure}
Now, with this setup constraints on the parameters are to be estimated, such
that the observational conditions of a spectral index $\alpha=2.5$ and
sufficiently high isotropy in the range $\inrange{E}{5}{30}{\exa\eV}$ are met.
Based on the results from \cref{sec:eval-defl} concerning the isotropies at
different source distances, only radii $\Robserver\ge\SI{2.4}{\mega\parsec}$
are considered in the following. Further the results of
$\Brms=\SI{20}{\nano\gauss}$ are dropped, due to being rather similar to the
other field strengths and thus not providing more insight.

Then, for each remaining radius and field strength the spectra \CRFlux~times
$E^{2.5}$ were plotted and by hand the intervals were chosen, in which the
resulting curves are approximately horizontal (\ie~correspond to the expected
spectral index). To fit the spectral index, in those intervals an
\emph{Orthogonal Distance Regression} was performed (using \texttt{scipy}'s
interface to \texttt{ODRPACK}), where the obtained uncertainties $\Delta{N}$
were provided as deviations in y-direction and $(\text{width of
bins})/\sqrt{12}$ as deviations in x-direction (according to the variance of
the rectangular distribution). Finally, the goodness of those fits is
quantified by their respective reduced $\chi^2$-test (the closer \redchisq~to
1, the better the fit):
\begin{gather*}
    \chi^2=\sum_{i=1}^{\text{No.~of samples}}
    \frac{\left(y_i-f(x_i)\right)^2}{\sigma_y^2} \\
    \redchisq=\frac{\chi^2}{\text{DoF}}
    \qc\text{DoF}=(\text{No.~of samples})-(\text{No.~of fitted parameters})
\end{gather*}

One might speculate -- while Fermi's Theory gives rise to a spectral index
$\alphasource=2$ from looking at one mechanism of acceleration -- that further
acceleration mechanisms come into play and that the true spectral index
deviates from the prediction. As an exemplary case, the collected data was also
reweighted to a source spectral index of $\alphasource=2.6$ and the resulting
spectra were discussed.

Additionally, for candidates with energies in the given intervals the
distributions of the deflection angles \ablenkung~were plotted and the
corresponding mean and standard deviation were computed as a simple measure to
estimate the isotropy.

\emph{Note:} For the observer spheres the object \texttt{ObserverLargeSphere}
was used, which only detects leaving particles (re-entering particles are
ignored).
Thus, the deflection angles -- which were computed as described in this section
-- fall into the interval $\inrange{\ablenkung}{0}{90}{\degree}$ (instead of
$\inrange{\ablenkung}{0}{180}{\degree}$, which is the expected behaviour).
This approach is marked as deprecated and it is instead recommended
to use \texttt{ObserverSurface(Sphere(center, radius))}, which detects all
particles regardless of their direction.

The resulting spectra are shown in \cref{fig:constraints}, the angular
distributions in \cref{fig:constrains-defl} and an overview over the chosen
energy intervals, the obtained spectral indices and the mean values of the
deflections are listed in \cref{tab:constr}

\subsubsection*{Results}
\begin{description}
    \item[General Behaviour]
        Most of the plotted spectra exhibit two kinks, which delimit an
        approx.~ horizontal plateau (marking the area of interest with
        $\alpha\sim2.5$).

        With increasing \Robserver~and fixed \Brms, those kinks are shifted to
        higher energies, while approx.~maintaining their relative (logarithmic)
        distance.

        Changing \Brms~while keeping \Robserver~fixed changes the overall shape
        of the spectra: the positions and relative distance of the kinks and
        the slopes of the curves below and above the plateau region.

    \item[Candidates for $\mathbf{\alphasource=2}$]
        The only case with a resulting spectral index whose standard deviation
        includes the expected value is $\Robserver=\SI{24}{\mega\parsec},
        \Brms=\SI{1}{\nano\gauss}$ with $\alphafalling=\num{2.55(6)}$. However
        the interval for which this value was obtained is somewhat too short
        ($\inrange{E}{2}{10}{\exa\eV}$).

        All other results yield spectral indices, which are too flat.

    \item[Candidates for $\mathbf{\alphasource=2.6}$]
        Increasing ${\alphasource}$ also increases all
        ${\alphafalling}$. This shifts all results for
        $\Robserver=\SI{2.4}{\mega\parsec}$ into the vicinity of the expected
        value (yet it is not included by any of the respective standard
        deviations). Out of these, $\Brms=\SI{10}{\nano\gauss}$ with
        $\alphafalling=\num{2.54(2)}$ is valid for an interval
        ($\inrange{E}{3}{40}{\exa\eV}$) which entirely includes the imposed
        constraints.

        The only other case being in line is $\Robserver=\SI{48}{\mega\parsec},
        \Brms=\SI{10}{\nano\gauss}$ with $\alphafalling=\num{2.48(8)}$,
        however its energy interval has a noticeably too large lower bound
        ($\inrange{E}{15}{100}{\exa\eV}$).

    \item[Notes on the Isotropy]
        The distributions of deflection angles in the respective energy
        intervals are \emph{for all cases} noticeable high, with rather regular
        semi-circle shapes (in y-log representation).

        The case $\Robserver=\SI{2.4}{\mega\parsec}, \Brms=\SI{1}{\nano\gauss}$
        with $\ablenkung=\SI{41.44(1942)}{\degree}$ exhibits the strongest
        deviation (by $7.9\%$) from the expected mean deflection angle for
        maximum isotropy $\ablenkung=\SI{45}{\degree}$. This is also reflected
        by the slight irregularities of the respective deflection angle
        distribution. However one can argue that the deviation of the mean
        value and the bumps in the distribution still fall within the
        acceptable regime.

\end{description}


\begin{landscape}

\begin{figure}[t]
    \begin{subfigure}[t]{\textwidth}
        \centering
        \def\source{-2.0}

\newcommand{\PlotOneConstraint}[3]{%
    \edef\temp{%
        \noexpand\addplot+ [%
            thick,
            error bars/.cd,
                y dir=both,y explicit,
                error bar style={thick},
        ] table [x=dE,y=N,y error=yerr]
            {constraints_spectrum-#1_brms-#3_r-#2.csv};
    }
    \temp
}

\newcommand{\PlotAllConstraints}[2]{%
    \pgfplotsforeachungrouped \bfield in {%
            1e-13,
            1e-12,
            5e-12} {%
        \PlotOneConstraint{#1}{#2}{\bfield}
    }
}


\tikzexternalenable
\tikzsetnextfilename{ext-constraints-fermi}

\begin{tikzpicture}
\begin{groupplot}[%
    group style={%
        % group name=energy loss plots,
        group size=3 by 1,
        xlabels at=edge bottom,
        ylabels at=edge left,
        xticklabels at=edge bottom,
        % yticklabels at=edge left,
    },
    % xmin=2e-2,xmax=2e3,
    xmode=log, ymode=log,
    xlabel={$E/\si{\eV}$},
    ylabel={$E^{2.5}\times{\CRFlux}/$a.u.},
    width=8.5cm,height=6.375cm,
    cycle list={[of colormap=Dark2]},
    grid=major,
    % title style={at={(.5,0)},below,yshift=-1em}
    title style={above,align=center}
]
    \nextgroupplot[%
        title={(a)~$\Robserver=\SI{2.4}{\mega\parsec}$}]
    \PlotAllConstraints{\source}{7e+22}

    \nextgroupplot[%
        title={(b)~$\Robserver=\SI{24}{\mega\parsec}$}]
    \PlotAllConstraints{\source}{7e+23}

    \nextgroupplot[%
        title={(c)~$\Robserver=\SI{48}{\mega\parsec}$},
        legend to name={ConstraintsLegend-fermi},
        legend style={legend columns=3,
            /tikz/every even column/.append style={column sep=2em}},
        legend cell align=left,
    ]
    \PlotAllConstraints{\source}{1e+24}

    \legend{%
        $\Brms=\SI{1}{\nano\gauss}$,
        $\Brms=\SI{20}{\nano\gauss}$,
        $\Brms=\SI{50}{\nano\gauss}$,
    }

\end{groupplot}

    \path (group c1r1.south east) --
        node[below,yshift=-3em]{\ref*{ConstraintsLegend-fermi}}
        (group c3r1.south west);

\end{tikzpicture}

\tikzexternaldisable

% vim: set ff=unix tw=79 sw=4 ts=4 et ic ai :

        \caption{Source spectral index $\alphasource=-2$, according to Fermi's theory}
        \label{fig:constraints-fermi}
        \vspace*{\baselineskip}
    \end{subfigure}

    \begin{subfigure}[t]{\textwidth}
        \centering
        \def\source{-2.6}

\newcommand{\PlotOneConstraintMod}[3]{%
    \edef\temp{%
        \noexpand\addplot+ [%
            thick,
            error bars/.cd,
                y dir=both,y explicit,
                error bar style={thick},
        ] table [x=dE,y=N,y error=yerr]
            {constraints_spectrum-#1_brms-#3_r-#2.csv};
    }
    \temp
}

\newcommand{\PlotAllConstraintsMod}[2]{%
    \pgfplotsforeachungrouped \bfield in {%
            1e-13,
            1e-12,
            5e-12} {%
        \PlotOneConstraintMod{#1}{#2}{\bfield}
    }
}


\tikzexternalenable
\tikzsetnextfilename{ext-constraints-mod}

\begin{tikzpicture}
\begin{groupplot}[%
    group style={%
        % group name=energy loss plots,
        group size=3 by 1,
        xlabels at=edge bottom,
        ylabels at=edge left,
        xticklabels at=edge bottom,
        % yticklabels at=edge left,
    },
    % xmin=2e-2,xmax=2e3,
    xmode=log, ymode=log,
    xlabel={$E/\si{\eV}$},
    ylabel={$E^{2.5}\times{\CRFlux}/$a.u.},
    width=8.5cm,height=6.375cm,
    cycle list={[of colormap=Dark2]},
    grid=major,
    % title style={at={(.5,0)},below,yshift=-1em}
    title style={above,align=center}
]
    \nextgroupplot[%
        title={(a)~$\Robserver=\SI{2.4}{\mega\parsec}$}]
    \PlotAllConstraintsMod{\source}{7e+22}

    \nextgroupplot[%
        title={(b)~$\Robserver=\SI{24}{\mega\parsec}$}]
    \PlotAllConstraintsMod{\source}{7e+23}

    \nextgroupplot[%
        title={(c)~$\Robserver=\SI{48}{\mega\parsec}$},
        legend to name={ConstraintsLegend-mod},
        legend style={legend columns=3,
            /tikz/every even column/.append style={column sep=2em}},
        legend cell align=left,
    ]
    \PlotAllConstraintsMod{\source}{1e+24}

    \legend{%
        $\Brms=\SI{1}{\nano\gauss}$,
        $\Brms=\SI{20}{\nano\gauss}$,
        $\Brms=\SI{50}{\nano\gauss}$,
    }

\end{groupplot}

    \path (group c1r1.south east) --
        node[below,yshift=-3em]{\ref*{ConstraintsLegend-mod}}
        (group c3r1.south west);

\end{tikzpicture}

\tikzexternaldisable

% vim: set ff=unix tw=79 sw=4 ts=4 et ic ai :

        \caption{Modified source spectral index $\alphasource=-2.6$}
        \label{fig:constraints-mod}
    \end{subfigure}
    \caption{Normalized flux spectra for two different source distributions}
    \label{fig:constraints}
\end{figure}

\begin{figure}[p]
    \centering
    \newcommand{\PlotOneConstrDeflection}[2]{%
    \edef\temp{%
        \noexpand\addplot+ [%
            const plot mark mid,
            thick,
        ] table [x=bins,y=N]
            {constraints_defl_brms-#2_r-#1.csv};
    }
    \temp
}

\newcommand{\PlotOneConstrSphere}[1]{%
    \pgfplotsforeachungrouped \bfield in {%
            1e-13,
            1e-12,
            5e-12} {%
        \PlotOneConstrDeflection{#1}{\bfield}
    }
}


\tikzexternalenable
\tikzsetnextfilename{ext-constraints-defl}

\begin{tikzpicture}
\begin{groupplot}[%
    group style={%
        % group name=energy loss plots,
        group size=3 by 1,
        xlabels at=edge bottom,
        ylabels at=edge left,
        % xticklabels at=edge bottom,
        yticklabels at=edge left,
        horizontal sep=2em,
        vertical sep=4em,
    },
    ymin=3e-4,
    % xmin=2e-2,xmax=2e3,
    xmode=linear, ymode=log,
    xlabel={$\ablenkung/\si{\degree}$},
    ylabel={relative abundance},
    width=8.5cm,height=6.375cm,
    cycle list={[of colormap=Dark2]},
    grid=major,
    % title style={at={(.5,0)},below,yshift=-1em}
    title style={above,align=center}
]
    \nextgroupplot[%
        title={(a)~$\Robserver=\SI{2.4}{\mega\parsec}$}]
    \PlotOneConstrSphere{7e+22}

    \nextgroupplot[%
        title={(b)~$\Robserver=\SI{24}{\mega\parsec}$}]
    \PlotOneConstrSphere{7e+23}

    \nextgroupplot[%
        title={(c)~$\Robserver=\SI{48}{\mega\parsec}$},
        legend to name={ConstrDeflLegend},
        legend style={legend columns=3,
            /tikz/every even column/.append style={column sep=2em}},
        legend cell align=left,
        % legend transposed=true,
    ]
    \PlotOneConstrSphere{1e+24}

    \legend{%
        $\Brms=\SI{1}{\nano\gauss}$,
        $\Brms=\SI{10}{\nano\gauss}$,
        $\Brms=\SI{50}{\nano\gauss}$,
    }

\end{groupplot}

    \path (group c1r1.south east) --
        node[below,yshift=-3em]{\ref*{ConstrDeflLegend}}
        (group c3r1.south west);

\end{tikzpicture}

\tikzexternaldisable

% vim: set ff=unix tw=79 sw=4 ts=4 et ic ai :

    \caption{Normalized distribution of deflection angles for the energy
        intervals given in \cref{tab:constr}}
    \label{fig:constrains-defl}
\end{figure}

\begin{table}[p]
    \centering
    \begin{tabular}{%
            *{4}{S}
            S[table-format=1.2,table-figures-uncertainty=1]
            S
            S[table-format=3.2,table-figures-uncertainty=1]
            S
            S[table-format=3.2,table-figures-uncertainty=1]
        }
        \toprule
        {\multirow{2}{*}{\Robserver/\si{\mega\parsec}}} &
        {\multirow{2}{*}{\Brms/\si{\nano\gauss}}} &
        {\multirow{2}{*}{\Emin/\si{\exa\eV}}} &
        {\multirow{2}{*}{\Emax/\si{\exa\eV}}} &
        \multicolumn{2}{c}{$\alphasource=2.0$} &
        \multicolumn{2}{c}{$\alphasource=2.6$} &
        {\multirow{2}{*}{\ablenkung/\si{\degree}}} \\
        \cmidrule{5-8}
        & & & & {\alphafalling} & {\redchisq} & {\alphafalling} & {\redchisq} & \\
        \midrule
        {\multirow{3}{*}{\num{2.4}}}
        & 1     & 1 & 3.5   & 2.02(8)  & .57   & 2.62(9)  & .74   & 41.44(1942)   \\
        & 10    & 3 & 40    & 1.93(3)  & 1.27  & 2.54(2)  & 1.18  & 43.72(1956)   \\
        & 50    & 8 & 150   & 1.97(2)  & 2.08  & 2.55(2)  & 1.80  & 44.49(1959)   \\
        \cmidrule{1-9}
%
        {\multirow{3}{*}{\num{24}}}
        & 1     & 2 & 10    & 2.55(6)  & .74   & 3.17(7)  & .88   & 44.01(1957)   \\
        & 10    & 7 & 100   & 2.37(4)  & 9.87  & 2.92(5)  & 11.28 & 44.54(1966)   \\
        & 50    & 30& 300   & 1.75(3)  & 1.35  & 2.22(2)  & 1.05  & 44.61(1966)   \\
        \cmidrule{1-9}
%
        {\multirow{3}{*}{\num{48}}}
        & 1     & 2 & 15    & 2.31(3)  & .99   & 2.93(3)  & 1.01  & 44.07(1949)   \\
        & 10    & 15& 100   & 2.02(7)  & 5.12  & 2.48(8)  & 5.21  & 44.35(1940)   \\
        & 50    & 20& 300   & 1.30(8)  & 25.00 & 1.68(11) & 52.96 & 44.53(1969)   \\
        \bottomrule
    \end{tabular}
    \caption{%
        For each radius and each field strength the chosen energy intervals,
        including the fitted spectral indices and averaged deflection angles in
        those intervals are listed. The spectral indices were fitted for two
        different source distributions and are accompanied by their respective
        reduced $\chi^2$-test.
    }
    \label{tab:constr}
\end{table}

\end{landscape}


% vim: set ff=unix tw=79 sw=4 ts=4 et ic ai :
