\section{Evaluation}
\subsection{The Monte-Carlo method}
\label{sec:eval-circle}
The exact value of the unit circle's surface area is
\begin{equation}
    A_{\mathrm{circle,exact}}=\eval{\pi\,r^2}_{r=1}=\pi=3.141\dotso
    \label{eq:circle-area-exact}
\end{equation}
While results of single MC runs may deviate rather noticeably,
plotting the histogram of all MC runs reveals a uniform distribution
centered around the exact value as expected from the central
limit theorem (see \cref{fig:circle-area-hist}).

\begin{figure}[ht]
    \centering
    \includegraphics[width=.75\textwidth]{circle-area-dist}
    \caption{Histogram of \num{1000} MC runs with mean and standard
    deviation \num{3.141(17)}. Notice the corresponding normal
distribution.}
    \label{fig:circle-area-hist}
\end{figure}

\subsection{Reweighting}
\label{sec:eval-rew}
One starts with computing the normalized histograms of the resulting spectra,
were the logarithm of the energies is used. The differential candidate number
$\dd{N}/\dd{E}$ can be approximated by dividing the number of particles for a
given bin by its width $\dd{E_i}=10^{\log(E_{i+1})-\log(E_{i})}$, where $E_{i},
E_{i+1}$ are the bin's edges; $\dd{N}/\dd{E}$ is plotted against the central
energies of the bins $10^{\log(E_{i+1})}+10^{\log(E_{i})}$ in
loglog-representation. The uncertainties are computed according to Poisson
statistics (\cref{eq:mc-std})
\begin{equation*}
    \Delta{y}=\frac{\dd{N}/\dd{E}}{\sqrt{N}}
\end{equation*}
were $N$ refers to the \emph{unweighted} and \emph{not normalized} number of
candidates per bin.

This is done for the energy spectra with $\alphaflat=-1$ and
$\alphafermi=-2$ and for the reweighted flat spectrum
($\alphaflat\to\alphafermi$, see \cref{eq:reweight}).

\begin{figure}[ht]
    \begin{subfigure}[b]{.75\textwidth}
        \centering
        % \input
        \caption{Fermi spectrum}
        \label{fig:rew-fermi}
    \end{subfigure}
    \begin{subfigure}[b]{.75\textwidth}
        \centering
        % \input
        \caption{flat spectrum}
        \label{fig:rew-flat}
    \end{subfigure}
    \begin{subfigure}[b]{.75\textwidth}
        \centering
        % \input
        \caption{reweighted flat spectrum}
        \label{fig:rew-rew}
    \end{subfigure}
    \caption{Normalized plots}
    \label{fig:rew}
\end{figure}

The histogram of the Fermi spectrum (see \cref{fig:rew-fermi}) clearly shows
the predominance of particles at lower energies and an approx.~linear (in
the loglog-representation, slope$\sim-1$) decrease of particle numbers with
increasing energy. This is also reflected in the increasing uncertainties,
which scale as $\Delta{y}\propto1/\sqrt{N}$.
While the spectral curve behaves approx.~as expected from first-order Fermi
acceleration, a deviation is noticeable when looking at the fitted spectral
index.
Also note the \enquote{plateau} at the end of the spectrum, which is due to
insufficient MC statistics in that area.

On the other hand, the flat spectrum (see \cref{fig:rew-flat}) exhibits a
uniform distribution of particle numbers across all energies and therefor
consistently small uncertainties. When applying the weight
$w_{E_0}=E_0^{1-2}=E_0^{-1}$ to this distribution, one obtains a spectrum,
which is in good agreement with the theory, while also maintaining the
well-behaved uncertainties (\cref{fig:rew-rew}).

\emph{Note:} Due to the clear advantages of this technique, all following
simulations will be performed with a flat spectrum and have their results
reweighted to correspond to Fermi-theory.

\todo{mention fluctuations when repeating simulations with unchanged
parameters?}


\subsection{Nuclear Interactions}
\todo{remove photo-disint for protons}
The influence of the individual interactions is analysed by looking at the
differential candidate number $\dd{N}/\dd{E}$ times $E^2$. Then, with a
log-scale on the x-axis, a power-law spectrum with $\alpha=-2$ appears as a
horizontal line, which is convenient for identifying and discussing deviations
from the expected curve.

$\dd{N}/\dd{E}$ is computed as described in \cref{sec:eval-rew} and $E$
describes the central bin values.

When looking at the respective plots, one immediately recognizes the most
important interactions for each setup:
\begin{description}
    \item[H close]
        \begin{itemize}
            \item photo-pion-prod.~with CMB
        \end{itemize}
    \item[H far]
        \begin{itemize}
            \item photo-pion-prod.~with CMB
            \item electron-pair-prod.~with CMB
        \end{itemize}
    \item[Fe close]
        \begin{itemize}
            \item photo-pion-prod.~with CMB
            \item photo-disint.~with CMB
        \end{itemize}
    \item[Fe far]
        \begin{itemize}
            \item photo-pion-prod.~with CMB
            \item photo-pion-prod.~with IRB
            \item photo-disint.~with CMB
            \item photo-disint.~with IRB
            \item electron-pair-prod.~with CMB
        \end{itemize}
\end{description}



\subsection{Magnetic Deflection}
\todo{compute and compare rigidities $\rho=r_L/l_c, r_L=E/ZeBc$}


\subsection{Constraints on the Sources}


% vim: set ff=unix tw=79 sw=4 ts=4 et ic ai :
