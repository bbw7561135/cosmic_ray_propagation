\section{Evaluation}
\subsection{The Monte-Carlo method}
\label{sec:eval-circle}
The exact value of the unit circle's surface area is
\begin{equation}
    A_{\mathrm{circle,exact}}=\eval{\pi\,r^2}_{r=1}=\pi=3.141\dotso
    \label{eq:circle-area-exact}
\end{equation}
While results of single MC runs may deviate rather noticeably,
plotting the histogram of all MC runs reveals a uniform distribution
centered around the exact value as expected from the central
limit theorem (see \cref{fig:circle-area-hist}).

\begin{figure}[ht]
    \centering
    \includegraphics[width=.75\textwidth]{circle-area-dist}
    \caption{Histogram of \num{1000} MC runs with mean and standard
    deviation \num{3.141(17)}. Notice the corresponding normal
distribution.}
    \label{fig:circle-area-hist}
\end{figure}

\subsection{Reweighting}
\label{sec:eval-rew}
One starts with computing the normalized histograms of the resulting spectra,
were the logarithm of the energies is used. The differential candidate number
$\dd{N}/\dd{E}$ can be approximated by dividing the number of particles for a
given bin by its width $\dd{E_i}=10^{\log(E_{i+1})-\log(E_{i})}$, where $E_{i},
E_{i+1}$ are the bin's edges; $\dd{N}/\dd{E}$ is plotted against the central
energies of the bins $\left(10^{\log(E_{i+1})}+10^{\log(E_{i})}\right)/2$ in
loglog-representation. The uncertainties are computed according to Poisson
statistics \cref{eq:mc-std}:
\begin{equation*}
    \Delta{y}=\frac{\dd{N}/\dd{E}}{\sqrt{N}}
\end{equation*}
were $N$ refers to the \emph{unweighted} and \emph{not normalized} number of
candidates per bin.

This is done for the energy spectra with $\alphaflat=-1$ and
$\alphafermi=-2$ and for the reweighted flat spectrum
($\alphaflat\to\alphafermi$, see \cref{eq:reweight}).

% \def\width{12.75cm}
% \def\height{9.56cm}
% \def\width{7.65cm}
% \def\height{5.74cm}
\def\width{6.8cm}
\def\height{5.1cm}
\def\binymin{3*10^-3}
\def\binymax{3}
\def\errymin{6*10^-25}
\def\errymax{6*10^-19}


\newcommand{\ReweightingAxis}[5]{%

\tikzexternalenable
\tikzsetnextfilename{ext-reweighting-#5}

\begin{tikzpicture}
    \pgfplotsset{set layers}

\begin{axis}[%
    width=\width, height=\height,
    scale only axis,
    axis y line*=left,
    axis x line=none,
    xmin={#3}, xmax={#4},
    ymin=\binymin, ymax=\binymax,
    ylabel={normalized number of events},
    ymode=log,
]
    \addplot [%
        ybar interval,
        ultra thin,
        fill=Paired-B,
        draw=Paired-B!20!white,
    ] table [x=bins,y=N]
        {reweighting-hist-alpha-#1-alpha_new-#2.csv};

\end{axis}

\begin{axis}[%
    width=\width, height=\height,
    scale only axis,
    axis y line*=right,
    xmin={#3}, xmax={#4},
    ymin=\errymin, ymax=\errymax,
    xlabel={$\log(E/\si{\eV})$},
    ylabel={$\dd{N}/\dd{E}/$a.u.},
    xtick distance=.5,
    grid=major,
    ymode=log,
    log basis y=10,
]
    \addplot [%
        draw=none,
        fill=none,
    ] table [%
        y={create col/linear regression={x=dE,y=dNdE}}
    ]
        {reweighting-errbar-alpha-#1-alpha_new-#2.csv};

    \addplot [%
        Paired-H,
        ultra thick,
        domain=18:22,
    ] {10^(\pgfplotstableregressiona*x+\pgfplotstableregressionb)};

    \addplot [%
        draw=none,
        error bars/.cd,
            y dir=both,y explicit,
            error bar style={ultra thick,color=black},
    ] table [x=dE,y=dNdE,y error=yerr]
        {reweighting-errbar-alpha-#1-alpha_new-#2.csv};

    \legend{,$\alpha_{\mathrm{fit}}=\pgfmathprintnumber{\pgfplotstableregressiona}$}
\end{axis}
\end{tikzpicture}

\tikzexternaldisable

}


% vim: set ff=unix tw=79 sw=4 ts=4 et ic ai :

\begin{figure}[b!]
    \centering
    \begin{subfigure}[b]{.75\textwidth}
        \centering
        \ReweightingAxis{-2.00}{None}{18.794}{21.54}{fermi}
        % \includegraphics[width=\textwidth]{extern/ext-reweighting-fermi}
        \caption{Fermi spectrum: direct simulation with $\alpha=-2$}
        \label{fig:rew-fermi}
        % \vspace*{\baselineskip}
    \end{subfigure}
\end{figure}
\begin{figure}[b!]
    \ContinuedFloat
    \centering
    \begin{subfigure}[b]{.75\textwidth}
        \centering
        \ReweightingAxis{-1.00}{None}{18.793}{21.795}{flat}
        % \includegraphics[width=\textwidth]{extern/ext-reweighting-flat}
        \caption{flat spectrum with $\alpha=-1$}
        \label{fig:rew-flat}
        % \vspace*{\baselineskip}
    \end{subfigure}
\end{figure}
\begin{figure}[t!]
    \ContinuedFloat
    \centering
    \begin{subfigure}[b]{.75\textwidth}
        \centering
        \ReweightingAxis{-1.00}{-2.00}{18.795}{21.793}{reweight}
        % \includegraphics[width=\textwidth]{extern/ext-reweighting-reweight}
        \caption{reweighted spectrum: simulated with $\alpha=-1$, then
            reweighted to $\alpha=-2$}
        \label{fig:rew-rew}
    \end{subfigure}

    \caption{Normalized histograms and differential energy spectra for
    different spectral indices}
    \label{fig:rew}
\end{figure}

The histogram of the Fermi spectrum (see \cref{fig:rew-fermi}) clearly shows
the predominance of particles at lower energies and an approx.~linear (in
the loglog-representation, slope $\sim-1$) decrease of particle numbers with
increasing energy. This is also reflected in the increasing uncertainties,
which scale as $\Delta{y}\propto1/\sqrt{N}$.
While the spectral curve behaves approx.~as expected from first-order Fermi
acceleration, a deviation is noticeable when looking at the fitted spectral
index.
Also note the \enquote{plateau} at the end of the spectrum, which is due to
insufficient MC statistics in that area.

On the other hand, the flat spectrum (see \cref{fig:rew-flat}) exhibits a
uniform distribution of particle numbers across all energies and therefor
consistently small uncertainties. When applying the weight
$w_{E_0}=E_0^{1-2}=E_0^{-1}$ to this distribution, one obtains a spectrum,
which is in good agreement with the theory, while also maintaining the
well-behaved uncertainties (\cref{fig:rew-rew}).

Due to the absence of any interactions, one would expect the observed spectrum
to be identical to the source spectrum, yet there are deviations in the fitted
spectral index. This is due to statistical effects from the MC simulations;
running these simulations multiple times would probably yield a normal
distribution around the \enquote{true} values. However these statistical
fluctuations are much more significant with the direct simulation (due to the
reasons discussed above), which additionally speaks in favor for the
reweighting.

\emph{Note:} Due to the clear advantages of this technique, all following
simulations will be performed with a flat spectrum and have their results
reweighted to correspond to Fermi-theory.


\subsection{Nuclear Interactions}
The influence of the individual interactions is analysed by looking at the
differential candidate number $\dd{N}/\dd{E}$ times $E^2$. Then, with a
log-scale on the x-axis, a power-law spectrum with $\alpha=-2$ appears as a
horizontal line, which is convenient for identifying and discussing deviations
from the expected curve.
$\dd{N}/\dd{E}$ and the uncertainties are computed as described in
\cref{sec:eval-rew} and $E$ describes the central bin values. The results
are depicted in \cref{fig:interactions}.

\begin{figure}[ht!]
    \centering
    \newcommand{\PlotOneInteraction}[3]{%
    \edef\temp{%
        \noexpand\addplot+ [%
            thick,
            error bars/.cd,
                y dir=both,y explicit,
                error bar style={thick},
        ] table [x=dE,y=N,y error=yerr]
            {interactions-#1-#2-#3.csv};
    }
    \temp
}

\newcommand{\PlotAllInteractions}[2]{%
    \PlotOneInteraction{#1}{#2}{free}
    % \addlegendimage{empty legend}
    \pgfplotsforeachungrouped \kind in {%
            Redshift,
            PhotoPionCMB,
            PhotoPionIRB,
            ElectronPairCMB,
            ElectronPairIRB,
            PhotoDisCMB,
            PhotoDisIRB} {%
        \PlotOneInteraction{#1}{#2}{\kind}
    }
}

\tikzexternalenable
\tikzsetnextfilename{ext-interactions}

\begin{tikzpicture}
\begin{groupplot}[%
    group style={%
        % group name=energy loss plots,
        group size=2 by 2,
        xlabels at=edge bottom,
        ylabels at=edge left,
        xticklabels at=edge bottom,
    },
    xmode=log, ymode=linear,
    xlabel={$E/\si{\eV}$},
    ylabel={$E^2\dd{N}/\dd{E}/$a.u.},
    width=8.5cm,height=6.375cm,
    cycle list={[of colormap=Dark2]},
    grid=major,
    % title style={at={(.5,0)},below,yshift=-1em}
    title style={at={(0,1)},above,align=left,xshift=3em}
]
    \nextgroupplot[xmin=7e17,xmax=7e21,title={(a) H close~~}]
        \PlotAllInteractions{H}{close}

    \nextgroupplot[xmin=7e17,xmax=2e23,title={(b) \ce{^{56}Fe} close}]
        \PlotAllInteractions{Fe56}{close}

    \nextgroupplot[xmin=7e17,xmax=7e21,title={(c) H far~~~~}]
        \PlotAllInteractions{H}{far}

    \nextgroupplot[xmin=7e17,xmax=2e23,
            title={(d) \ce{^{56}Fe} far~~},
            legend to name={InteractionLegend},
            legend style={legend columns=2,
                /tikz/every even column/.append style={column sep=1.05cm}},
            legend cell align=left,
            legend transposed=true,
        ]
        \PlotAllInteractions{Fe56}{far}

    \legend{%
        free,Redshift,
        PhotoPionCMB,PhotoPionIRB,
        PhotoPairCMB,PhotoPairIRB,
        % ElectronPairCMB,ElectronPairIRB,
        PhotoDisCMB,PhotoDisIRB,
    }

\end{groupplot}

    \path (group c1r2.south east) --
        node[below,yshift=-1.2cm]{\ref*{InteractionLegend}}
        (group c2r2.south west);

\end{tikzpicture}

\tikzexternaldisable

% vim: set ff=unix tw=79 sw=4 ts=4 et ic ai :

    % \includegraphics{extern/ext-interactions}
    \caption{Normalized energy spectra $\dd{N}/\dd{E}\times{E^2}$ vs.~$E$ for
        all cases}
    \label{fig:interactions}
\end{figure}

When looking at the respective plots, there are a few features to recognize:
\begin{description}
    \item[H close]
        All curves are nearly horizontal for $E<\SI{3e20}{\electronvolt}$
        corresponding to an (unchanged) spectral index $\alpha\approx-2$,
        \ie~no significant energy loss took place. Fluctuations are of
        statistical nature.

        At higher energies, \emph{Photo-Pion-Prod.~(CMB)} causes the
        spectrum first to flatten before steeply falling down at
        $E>\SI{1e21}{\electronvolt}$ (with $\alphafalling\approx-2.31$). The CR
        flux is shifted to lower energies.

        The cutoff appears at higher energies than the estimated
        \SI{5e19}{\electronvolt}, which is no surprise here, because of the
        close source distances \SIrange{0}{10}{\mega\parsec}.

    \item[H far]
        The peak of shifted flux due to \emph{Photo-Pion-Prod.~(CMB)} is formed
        more sharply ($\alphafalling\approx-5.58$), with no
        detected candidates beyond some maximum energy. However that maximum
        value lies with $\sim\SI{5e20}{\electronvolt}$ significantly further
        right than expected.

        The curve of \emph{Photo-Pair-Prod.~(CMB)} shows a weaker but
        noticeable decrease of flux without cutoff (with
        $\alphafalling\approx-2.22$). While this behaviour is
        qualitatively expected, the starting point of this effect lies with
        $\sim\SI{2e19}{\electronvolt}$ also about one order of magnitude too
        far right.

    \item[Fe close]
        The cutoff due to \emph{Photo-Pion-Prod.~(CMB)} is also present (with
        $\alphafalling\approx-2.96$) and
        lies about two orders of magnitude further right compared to hydrogen,
        which is a reasonable result, considering the higher number of
        nucleons on which the total energy is distributed. In addition to
        the peak of left-shifted flux immediately before the cutoff,
        another peak in the spectrum seems to emerge in the region
        \SIrange{1e20}{1e22}{\electronvolt}, which might be attributed to
        secondary particles resulting from \emph{Nuclear Decay} after the
        nucleus was converted to an unstable nuclide during
        Photo-Pion-Prod.~(\eg~${\text{p}+\gamma\to\text{n}+\pi^0}$ yielding
        \ce{^{56}Co} with a half-life of around 77 days; see also
        \cref{tab:secondaries}).

        This hypothesis could be supported by the curve of
        \emph{Photo-Disint.~(CMB)}, which exhibits a rather steep spectrum
        at $\sim\SI{5e21}{\electronvolt}$ resulting in a noticeable decrease
        of flux at energies above that value (with
        $\alphafalling\approx-2.89$). Additionally a very prominent peak is
        apparent for lower energies. This flattening of the spectrum
        is probably caused by secondaries and products of nuclear
        decay with left-shifted energies.

    \item[Fe far]
        The aforementioned weak second peak in the curve of
        \emph{Photo-Pion-Prod.~(CMB)} is now present more distinctly
        ($\alphafalling\approx-2.69$), while the first peak disappeared and the
        cutoff remains only in form of non-existent detections beyond some
        maximum energy, with the steep decline having completely disappeared.

        \emph{Photo-Pion-Prod.~(IRB)} also shows a peak at a similar
        location as with the CMB, followed by a slightly steeper spectrum,
        compared to the initial Fermi spectrum ($\alphafalling\approx-2.12$).
        No cutoff is apparent.

        \emph{Photo-Disint.~(CMB)} now exhibits steep descent
        ($\alphafalling\approx-3.40$) with a clear cutoff at
        $\sim\SI{2e21}{\electronvolt}$, meaning all \Fe-nuclei with larger
        energy have disintegrated or decayed into secondaries, causing the
        strong left-shifted peak in the spectrum.

        While \emph{Photo-Disint.~(IRB)} also shows a strong left-shifted
        peak reflecting the production of secondaries, no cutoff is present
        and the spectrum seems to return the initial spectral index after a
        steeper decline with $\alphafalling\approx-2.45$.

        When given sufficient propagation time (with longer distances),
        energy decay due to \emph{Photo-Pair-Prod.~(CMB)} becomes also
        noticeable for nuclei with energies greater than
        $\SI{1e21}{\electronvolt}$, apparent from a decline of the spectrum
        with $\alphafalling\approx-3.63$. Since this process does not change
        the internal structure of the nuclei, no unstable products are produced
        and thus no decay happens. This is reflected by a missing peak of
        left-shifted flux in the spectrum.

\end{description}

In order to quantify the effects of the interaction mechanisms somewhat
more thoroughly and to compare them with each other, one can estimate a
\emph{loss time scale} by looking at the maximum energy loss and multiplying
the square root of the ratio of the corresponding mass and that energy
difference with some scale length:
\begin{equation}
    \tau\sim d_{\mathrm{scale}}\times\sqrt{\frac{m}{E_0-E_{\mathrm{final}}}}.
\end{equation}
This expresses the time during which the candidate experiences the given energy
loss (the shorter the time, the more significant the respective mechanism). In
order to obtain comparable results, for all distance regimes the same
scale length $d_{\mathrm{scale}}=\SI{100}{\mega\parsec}$ was chosen (see
\cref{tab:loss-time}).

\begin{table}[ht]
    \centering
    \sisetup{table-parse-only,table-column-width=10ex}
    \begin{tabular}{lSSSS}
        \toprule
        \multirow{2}{*}{Interaction}
        & \multicolumn{2}{c}{H} & \multicolumn{2}{c}{\Fe} \\
        \cmidrule{2-5}
        & {close} & {far} & {close} & {far} \\
        \midrule
        Redshift            & 2748  & 293   & 4000  & 428   \\
        Pion-Prod.~(CMB)    & 133   & 130   & 25    & 25    \\
        Pion-Prod.~(IRB)    & 192   & 135   & 25    & 25    \\
        Pair-Prod.~(CMB)    & 2494  & 221   & 861   & 185   \\
        Pair-Prod.~(IRB)    & 89419 & 8964  & 33313 & 2920  \\
        Photo-Disint.~(CMB) & {inf} & {inf} & 25    & 25    \\
        Photo-Disint.~(IRB) & {inf} & {inf} & 29    & 26    \\
        \bottomrule
    \end{tabular}
    \caption{Energy loss time scales for all cases in years}
    \label{tab:loss-time}
\end{table}

\begin{description}
    \item[Redshift]
        Energy loss due to the cosmological redshift of the source of course
        depends on the source's distance: for both candidate types the loss
        time of the far regime is one order of magnitude smaller than that of
        the close regime, which is reasonable.

        Further, this mechanism is
        independent of the particle's energy, and thus does not change the
        spectral index.

    \item[Pion-Production]
        As a proton with sufficient energy is estimated to loss all of its energy
        after $\sim\SI{30}{\mega\parsec}$, which is of the same order of
        magnitude as the close distance regime, the similar loss times for both
        distance regimes are reasonable.

        Since $\tau\propto\sqrt{m}$, one might expect longer iron time scales,
        however the opposite is the case. This could probably be attributed
        to the higher number of nucleons, which can undergo this process and
        additional nuclear decay of unstable secondaries (see also
        \cref{tab:secondaries}).

        While the IRB interaction is comparable to the CMB interaction in all
        cases, it is more significant for hydrogen on larger distances. This
        might be due to a lower photon density and a therewith smaller cross
        section. This is again compensated by iron's higher nucleon number.

    \item[Pair-Production]
        As expected from the smaller fractional energy loss and as already seen
        in \cref{fig:interactions}, pair production makes itself noticeable not
        until larger distances.

        While comparable on the large distance regime, the shorter time scale
        for iron on the close regime might again be explained by its higher cross
        section.

        While for both candidate types the time scales in interaction with the
        IRB are rather large (and thus less significant), their difference is
        probably also due to the lower photon density and iron's higher cross
        section.

    \item[Photo-Disintegration]
        The infinite time scales for hydrogen show that no energy loss
        occurred, since there is no nucleus which can disintegrate into smaller
        nuclei (as expected from theory; this simulation could have been
        omitted).

        The values for iron are comparable to energy loss due to
        pion-production and thus belong to the more significant processes (as
        already seen in \cref{fig:interactions}). Interaction with IRB takes
        slightly longer on close distances.
\end{description}

\begin{table}[p]
    \centering
    \newcolumntype{E}{>{\zzz}l<{\relax}@{}}
    \def\zzz#1\relax{\ce{#1}}
    \begin{tabular}{lESES}
        \toprule
        \multirow{2}{*}{Interaction}
        & \multicolumn{2}{c}{close} & \multicolumn{2}{c}{far} \\
        \cmidrule{2-5}
        & {Product} & {~Abundance/\%} & {Product} & {~Abundance/\%} \\
        \midrule
%
        \multirow{3}{*}{\shortstack[l]{%
            Pion-Prod.~(CMB)\\
            \footnotesize
            close: 16.43\% decayed \\
            \footnotesize
            far: 23.03\% decayed
        }}
        & ^1H & 79.38       & ^1H & 99.37 \\
        & {n} & 15.38         & {traces} & 0.63 \\
        & {traces} & 5.24   & & \\
        \cmidrule{1-5}
%
        \multirow{7}{*}{\shortstack[l]{%
            Pion-Prod.~(IRB) \\
            \footnotesize
            close: 3.03\% decayed \\
            \footnotesize
            far: 42.16\% decayed
        }}
        & ^1H & 48.49       & ^1H & 90.86 \\
        & ^{55}Fe^* & 23.05   & {traces} & 9.14 \\
        & ^{55}Mn & 21.46   & & \\
        & {n} & 3.34          & & \\
        & ^{54}Mn^* & 2.23    & & \\
        & ^{54}Fe & 1.11    & & \\
        & {traces} & 0.32   & & \\
        \cmidrule{1-5}
%
        \multirow{5}{*}{\shortstack[l]{%
            Photo-Disint.~(CMB) \\
            \footnotesize
            close: 39.35\% decayed \\
            \footnotesize
            far: 44.86\% decayed
        }}
        & ^1H & 87.73       & ^1H & 99.01 \\
        & {n} & 5.88          & {traces} & 0.99 \\
        & ^4He & 2.12       & & \\
        & ^3He & 1.89       & & \\
        & {traces} & 2.38   & & \\
        \cmidrule{1-5}
%
        \multirow{9}{*}{\shortstack[l]{%
            Photo-Disint.~(IRB) \\
            \footnotesize
            close: 6.28\% decayed \\
            \footnotesize
            far: 59.02\% decayed
        }}
        & ^1H & 62.67       & ^1H & 89.17 \\
        & ^{55}Fe^* & 16.36   & ^4He & 3.02 \\
        & ^{54}Fe & 4.89    & ^3He & 1.82 \\
        & ^{53}Mn^* & 3.58    & {traces} & 6.00 \\
        & ^{55}Mn & 3.52    & & \\
        & ^{54}Mn^* & 3.07    & & \\
        & ^{52}Cr & 1.76    & & \\
        & {n} & 1.02        & & \\
        & {traces} & 3.13   & & \\
        \bottomrule
    \end{tabular}
    \caption[Secondaries of \Fe]{%
    \textbf{Secondaries of \Fe:}
    The decay of an iron nucleus may be triggered by Photo-Pion-Production and
    Photo-Disintegration.

    For each simulation \num{10000} candidates were propagated, of which the
    specified percentage decayed. The isotope designations of the decay
    products were determined from the recorded candidate IDs and their
    composition is listed here. Radioactive isotopes are indicated by an
    asterisk.

    Note the absence of neutrons in the far regime: this is due to free
    neutrons being unstable with a half life of \SI{881.5(15)}{\second}.}
    \label{tab:secondaries}
\end{table}



\subsection{Magnetic Deflection}
\todo{%
compute and compare rigidities $\rho=r_L/l_c, r_L=E/ZeBc$

plot cr spectrum $\dd{N}/\dd{\rho}, \rho=\Lamor/l_c$

mean and std of deflection (note: only leaving particles are detected)

mean and std of trajectory length

incl.~distributions for above two

effect of B on the energy spectrum

changes of mean deflection based on B, d (upper limit?)

changes of mean trajectory based on B, d (physics based upper limit? -> c, age
of the universe)

in what range can a UHECR source produce an isotropic distribution at the
earth?
}

\begin{table}
    \centering
    \begin{tabular}{SSSSS}
        \toprule
        {\multirow{2}{*}{\Robserver/\si{\mega\parsec}}} &
        {\multirow{2}{*}{\Brms/\si{\nano\gauss}}} &
        \multicolumn{3}{c}{Measurements} \\
        \cmidrule{3-5}
        & & {\alphafalling} & {\traj/\si{\mega\parsec}} & {\ablenkung/\si{\degree}}
        \\
        \midrule
        {\multirow{4}{*}{\num{24e-3}}}
        & 1     & -2.00 & 0.03(26)      & 0.49(161)     \\
        & 10    & -2.04 & 0.17(619)     & 08.16(1845)   \\
        & 20    & -3.07 & 0.23(420)     & 20.46(3242)   \\
        & 50    & -3.08 & 00.88(3366)   & 48.93(4602)   \\
        \cmidrule{1-5}
%
        {\multirow{4}{*}{\num{.24}}}
        & 1     & -2.22 & 01.70(5405)   & 10.24(2037)   \\
        & 10    & -2.31 & 019.75(16619) & 49.20(4726)   \\
        & 20    & -2.29 & 026.88(21741) & 59.24(4756)   \\
        & 50    & -2.35 & 040.78(21228) & 73.38(4535)   \\
        \cmidrule{1-5}
%
        {\multirow{4}{*}{\num{.48}}}
        & 1     & -2.47 & 006.89(10490) & 20.44(3270)   \\
        & 10    & -2.38 & 055.17(29546) & 63.12(4727)   \\
        & 20    & -2.39 & 071.12(31868) & 72.39(4562)   \\
        & 50    & -2.40 & 128.14(46309) & 82.15(4249)   \\
        \cmidrule{1-5}
%
        {\multirow{4}{*}{\num{2.4}}}
        & 1     & -3.03 & 106.87(41466) & 50.54(4807)   \\
        & 10    & -2.99 & 362.74(71616) & 83.49(4188)   \\
        & 20    & -3.00 & 419.58(77142) & 86.59(4094)   \\
        & 50    & -2.97 & 526.08(86515) & 88.17(3956)   \\
        \cmidrule{1-5}
%
        {\multirow{4}{*}{\num{24}}}
        & 1     & -3.30 & 1361.81(125738) & 80.35(4461) \\
        & 10    & -3.35 & 2378.69(130089) & 88.25(3975) \\
        & 20    & -3.35 & 2527.83(127112) & 89.70(3929) \\
        & 50    & -3.29 & 2669.64(127207) & 89.65(3924) \\
        \cmidrule{1-5}
%
        {\multirow{4}{*}{\num{48}}}
        & 1     & -3.23 & 2006.20(145893) & 80.25(4411) \\
        & 10    & -3.18 & 2646.78(141007) & 86.68(3996) \\
        & 20    & -3.22 & 2709.92(140225) & 88.53(3928) \\
        & 50    & -3.37 & 2726.14(135104) & 89.52(3906) \\
        \bottomrule
    \end{tabular}
\end{table}



\subsection{Constraints on the Sources}


% vim: set ff=unix tw=79 sw=4 ts=4 et ic ai :
