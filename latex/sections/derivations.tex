\section{Derivation of the power-law spectrum}
\label{sec:app1}
From Fermi's theory, one has from the energy gain per encounter of some test
particle with a shock front $\Delta{E}=\zeta{E}$ the total energy after $n$
encounters $E_n=E_0(1+\zeta)^n$ for a given initial energy $E_0$. This
expression can be rearranged to yield the number of encounters required to
reach some energy $E$:
\begin{equation}
    n={\log\left(\frac{E}{E_0}\right)}/{\log(1+\zeta)}
    \label{eq:app-num-enc}
\end{equation}

It can be easily shown, that the probability to remain in the
vicinity of the accelerating shock is $(1-\Pesc)^n$, where \Pesc~denotes the
escape probability per encounter.
Then the number of particles accelerated to energies greater than $E$ is
proportional to the sum of probabilities to remain, starting at the minimal
energy after $n$ encounters and taking into account all higher orders:
\begin{equation}
    N(>E)\propto\sum_{m=n}^{\infty}(1-\Pesc)^m,
    \label{eq:NgreaterE}
\end{equation}
which can be split into two geometric sums, whose limits can be evaluated via
\begin{gather*}
    \sum_{k=0}^{n}x^k=\frac{1-x^{n+1}}{1-x}
    \quad\underset{n\to\infty}{\longrightarrow}\quad
    \frac{1}{1-x}\qc\abs{x}<1
\end{gather*}
with some more algebra, one finds
\begin{align*}
    \sum_{m=n}^{\infty}(1-\Pesc)^n
    &=\sum_{n=0}^{\infty}(1-\Pesc)^n-\sum_{m=0}^{n-1}(1-\Pesc)^m \\
    &=\frac{1}{\Pesc}-\frac{1-(1-\Pesc)^n}{\Pesc}
    =-\frac{(1-\Pesc)^n}{\Pesc}
\end{align*}
and using \cref{eq:app-num-enc}
\begin{align*}
    (1-\Pesc)^n
    &=\exp\left[\frac{\log\left(\frac{E}{E_0}\right)}{\log(1+\zeta)}\log(1-\Pesc)\right]\\
    &=\left(\frac{E}{E_0}\right)^{{\log(1-\Pesc)}/{\log(1+\zeta)}}
    =\left(\frac{E}{E_0}\right)^{-\alpha}
\end{align*}
one arrives at
\begin{equation}
    \boxed{%
        N(>E)\propto\frac{1}{\Pesc}\left(\frac{E}{E_0}\right)^{-\alpha}}
\end{equation}
Here the spectral index $\alpha$ has been introduced:
\begin{align*}
    \alpha&=\frac{-\log(1-\Pesc)}{\log(1+\zeta)}
    =\frac{\log\left(\frac{1}{1-\Pesc}\right)}{\log(1+\zeta)}\\
    &\approx\frac{\frac{1}{1-\Pesc}-1}{\zeta}\approx\frac{\Pesc}{\zeta}
\end{align*}
with the approximations being valid for $\Pesc,\zeta\ll1$, which allow to use
\begin{equation*}
    \log(1+x)\approx x\qc x\ll1
\end{equation*}
and
\begin{equation*}
    \frac{1}{1-\Pesc}-1=\frac{1-1+\Pesc}{1-\Pesc}=\frac{\Pesc}{1-\Pesc}\approx\Pesc.
\end{equation*}


\section{Derivation of the compression ratio}
\label{sec:app2}
The aim of the following consideration is to show
\begin{equation}
    (r+1)\left[r^2\frac{2-\gamma}{\MAsq}+r\left(\frac{\gamma}{\MAsq}+\frac{2}{\Mcsq}+\gamma-1\right)-\left(\gamma+1\right)\right]=0
\end{equation}
with the \emph{Alfven} and \emph{Sonic Mach Numbers} (in terms of upstream
quantities)
\begin{gather}
    \MA=\frac{u_1}{c_{\mathrm{Alfven}}}=\frac{u_1}{B_1/\sqrt{4\pi\rho_1}}
    \label{eq:ma} \\
    \Mc=\frac{u_1}{c_{s}}=\frac{u_1}{\sqrt{\gamma p_1/\rho_1}}
    \label{eq:ms}
\end{gather}
in the context of ideal Magnetohydrodynamics
($\vec{E}+\vec{u}\times\vec{B}=0$):
\begin{subequations}
\begin{align}
    \pt\rho=-\Div(\rho\vec{u}) & \quad\text{(cons.~of mass)} \\
    \pt\left(\rho\vec{u}\right)=-\Div \left[
        \rho\vec{u}\vec{u}+\left(p+\frac{B^2}{8\pi}\right)\mathds{1}-\frac{\vec{B}\vec{B}}{4\pi}\right]
    & \quad\text{(cons.~of momentum)} \\
    \pt\left(\frac{\rho{u^2}}{2}+\frac{B^2}{8\pi}+\frac{p}{\gamma-1}\right)
    =-\Div\left[\left(\frac{\rho{u^2}}{2}+\frac{\gamma}{\gamma-1}p\right)\vec{u}-\frac{\left(\vec{u}\times\vec{B}\right)\times\vec{B}}{4\pi}\right]
    & \quad\text{(cons.~of energy)}
\end{align}
\end{subequations}

Now when considering shock fronts, quantities indicated with
\emph{1}/\emph{2} refer to the upstream/downstream region and
\emph{n}/\emph{t} refer to the components normal/transversal to the shock.
In particular, one has:
\begin{equation*}
    \vec{X}\cdot\vec{n}=X_n\qc\pdv{\bullet}{n}=\vec{n}\cdot\nabla(\bullet),
\end{equation*}
where $\vec{n}$ is the normal vector of the shock front.

Since shocks pose discontinuities, the value of some quantity jumps across
the front, which is expressed by
\begin{equation*}
    \llbracket X \rrbracket=X_1-X_2.
\end{equation*}
Further, when transforming into the rest-frame of the shock, all quantities
can be considered static, \ie~$\pt\bullet=0$.

The MHD equations thus become the \emph{Rankine-Hugoniot relations}:
\begin{subequations}
\begin{align}
    \llbracket\rho u_n\rrbracket=0 & \quad\text{(cons.~of mass normal to
    shock)} \\
    \left\llbracket\rho{u_n}\vec{u}+\left(p+\frac{B^2}{8\pi}\right)\vec{n}-\frac{\vec{B}B_n}{4\pi}\right\rrbracket=0
    & \nonumber \\
    \implies
        \left\llbracket\rho{u_n^2}+p+\frac{B^2}{8\pi}\right\rrbracket=0
        & \quad\text{(cons.~of momentum normal to shock)} \\
    \implies
        \left\llbracket\rho{u_n}\vec{u}_t-\frac{B_n\vec{B}_t}{4\pi}\right\rrbracket=0
        & \quad\text{(cons.~of momentum transversal to shock)} \\
    \left\llbracket\left(\frac{\rho{u^2}}{2}+\frac{\gamma}{\gamma-1}p\right)u_n+\frac{B^2}{4\pi}u_n-\vec{u}\cdot\vec{B}\frac{B_n}{4\pi}\right\rrbracket=0
    & \quad\text{(cons.~of energy)}
\end{align}
\end{subequations}

Additional jump conditions come from the magnetic field being
divergence-free:
\begin{equation}
    \Div\vec{B}=0\implies\llbracket{B_n}\rrbracket=0
\end{equation}
and from the law of induction:
\begin{equation*}
    \pt\vec{B}+\nabla\times\vec{E}=\pt\vec{B}+\nabla\times(\vec{B}\times\vec{u})=0
\end{equation*}
with vanishing time derivatives, integrating over some volume and applying
Stoke's theorem:
\begin{align*}
    \int_V\nabla\times\left(\vec{B}\times\vec{u}\right)\dd{V}
    =\int_{\partial{V}}\vec{n}\times\left(\vec{B}\times\vec{u}\right)\dd{S}
    =\int_{\partial{V}}\left[(\vec{n}\cdot{\vec{u}})\vec{B}-(\vec{n}\cdot\vec{B})\vec{u}\right]\dd{S}
    =\int_{\partial{V}}\left[v_n\vec{B}-B_n\vec{u}\right]\dd{S}=0
\end{align*}
\begin{equation}
    \implies \llbracket v_n B_t-B_n v_t \rrbracket = 0
\end{equation}
(where obviously $u_n B_n - B_n u_n = 0$).

Here exemplarily a situation with the upstream magnetic field perpendicular
and the upstream flow parallel to the shock normal is considered, \ie:
\begin{gather*}
    \vec{B}_1\perp\vec{n}\implies B_n=0 \qc
    \vec{u}_1\parallel\vec{n}\implies u_t=0.
\end{gather*}

The aim is now to find an expression for the \emph{compression ratio}
\begin{equation}
    r=\frac{\rho_2}{\rho_1}
    \label{eq:r}
\end{equation}
in terms of upstream quantities, using (summarizing the jump conditions for
the particular setup):%
\begin{subequations}
\begin{align}
    \rho_1 u_1&=\rho_2 u_2 \label{eq:a1} \\
    \rho_1 u_1^2+p_1+\frac{B_1^2}{8\pi}&=\rho_2
    u_2^2+p_2+\frac{B_2^2}{8\pi} \label{eq:a2} \\
    \left(\frac{u_1^2}{2}+\frac{\gamma}{\gamma-1}\frac{p_1}{\rho_1}\right)\rho_1u_1+\frac{B_1^2}{4\pi}u_1
    &=\left(\frac{u_2^2}{2}+\frac{\gamma}{\gamma-1}\frac{p_2}{\rho_2}\right)\rho_2u_2+\frac{B_2^2}{4\pi}u_2
    \label{eq:a3} \\
    u_1B_1&=u_2B_2 \label{eq:a4}
\end{align}
\end{subequations}

\textbf{\Cref{eq:r,eq:a1,eq:a4}:}
\begin{equation}
    r=\frac{\rho_2}{\rho_1}=\frac{u_1}{u_2}
    \implies\rho_2=r\rho_1\qc u_2=\frac{u_1}{r}\qc B_2=r B_1
    \label{eq:a1.1}
\end{equation}

\textbf{\Cref{eq:a2}:}
\begin{gather*}
    \left(1+\frac{p_1}{\rho_1u_1^2}+\frac{1}{2}\frac{B_1^2}{4\pi\rho_1u_1^2}\right)\rho_1u_1^2=
    \left(1+\frac{p_2}{\rho_2u_2^2}+\frac{1}{2}\frac{B_2^2}{4\pi\rho_2u_2^2}\right)\rho_2u_2^2
\end{gather*}
with \cref{eq:ma,eq:ms,eq:a1.1} one has
\begin{gather*}
    \frac{1}{\gamma}\frac{\gamma{p_1}}{\rho_1u_1^2}=\frac{1}{\gamma\Mcsq}\qc
    \frac{B_1^2}{4\pi\rho_1u_1^2}=\frac{1}{\MAsq}\\
    \frac{1}{\gamma}\frac{\gamma{p_2}}{\rho_2u_2^2}=\frac{c_2^2}{\gamma{u_2^2}}\qc
    \frac{B_2^2}{4\pi\rho_2u_2^2}=r^3\frac{B_1^2}{4\pi\rho_1u_1^2}=\frac{r^3}{\MAsq}\qc
    \rho_2u_2^2=\frac{\rho_1u_1^2}{r}
\end{gather*}
which simplifies above expression to
\begin{gather*}
    \implies
    \left(1+\frac{1}{\gamma\Mcsq}+\frac{1}{2\MAsq}\right)
    =\left(1+\frac{c_2^2}{\gamma{u_2^2}}\right)\frac{1}{r}+\frac{r^2}{2\MAsq}
    \\ \implies
    \gamma\left[r\left(1+\frac{1}{\gamma\Mcsq}+\frac{1-r^2}{2\MAsq}\right)-1\right]=\frac{c_2^2}{u_2^2}
    \numberthis\label{eq:a2.1}
\end{gather*}

\textbf{\Cref{eq:a3}:}
\begin{gather*}
    \left(\frac{1}{2}+\frac{1}{\gamma-1}\frac{\gamma{p_1}}{\rho_1u_1^2}+\frac{B_1^2}{4\pi\rho_1u_1^2}\right)\rho_1u_1^3=
    \left(\frac{1}{2}+\frac{1}{\gamma-1}\frac{\gamma{p_2}}{\rho_2u_2^2}+\frac{B_2^2}{4\pi\rho_2u_2^2}\right)\rho_2u_2^3
\end{gather*}
using the same simplifications as above:
\begin{gather*}
    \implies
    \left(\frac{1}{2}+\frac{1}{\gamma-1}\frac{1}{\Mcsq}+\frac{1}{\MAsq}\right)
    =\left(\frac{1}{2}+\frac{1}{\gamma-1}\frac{c_2^2}{u_2^2}\right)\frac{1}{r^2}+\frac{r}{\MAsq}
    \\ \implies
    (\gamma-1)\left[r^2\left(\frac{1}{2}+\frac{1}{\gamma-1}\frac{1}{\Mcsq}+\frac{1-r}{\MAsq}\right)-\frac{1}{2}\right]=\frac{c_2^2}{u_2^2}
    \numberthis\label{eq:a3.1}
\end{gather*}

\textbf{Comparing \Cref{eq:a2.1,eq:a3.1}:}
\begin{gather*}
    \gamma\left[r\left(1+\frac{1}{\gamma\Mcsq}+\frac{1-r^2}{2\MAsq}\right)-1\right]=
    (\gamma-1)\left[r^2\left(\frac{1}{2}+\frac{1}{\gamma-1}\frac{1}{\Mcsq}+\frac{1-r}{\MAsq}\right)-\frac{1}{2}\right]
\end{gather*}
sorting by powers of $r$:
\begin{gather*}
    r^3\left(-\frac{\gamma}{2\MAsq}+\frac{\gamma-1}{\MAsq}\right)
    +r^2\left(-\frac{\gamma-1}{2}-\frac{1}{\Mcsq}-\frac{\gamma-1}{\MAsq}\right)
    +r\left(\gamma+\frac{1}{\Mcsq}+\frac{\gamma}{2\MAsq}\right)
    -\gamma+\frac{\gamma-1}{2}=0
\end{gather*}
multiplying by two, flipping the sign and regrouping the terms:
\begin{gather*}
    r^3\frac{2-\gamma}{\MAsq}+r^2\left(\gamma-1+\frac{2}{\Mcsq}+\frac{2\gamma-2}{\MAsq}\right)
    -r\left(2\gamma+\frac{2}{\Mcsq}+\frac{\gamma}{\MAsq}\right)+(\gamma+1)=\\
    \frac{2-\gamma}{\MAsq}\left(r^3-r^2\right)
    +\left(\gamma-1+\frac{2}{\Mcsq}+\frac{\gamma}{\MAsq}\right)\left(r^2-r\right)
    -(\gamma+1)(r-1)=0 \\
    \implies\boxed{%
        (r-1)\left[r^2\frac{2-\gamma}{\MAsq}+r\left(\gamma-1+\frac{2}{\Mcsq}+\frac{\gamma}{\MAsq}\right)-(\gamma+1)\right]=0
    }
\end{gather*}
which is exactly, what was to be shown.

In case of large upstream energies (\ie~large $u_1$), one has $\MA,\Mc\gg1$
and the terms $\propto1/\MAsq,1/\Mcsq$ become negligible. The non-trivial
solution of above expression $r\ne1$ then simplifies to:
\begin{equation}
    r(\gamma-1)-(\gamma+1)=0
    \implies\boxed{r=\frac{\gamma+1}{\gamma-1}}
    \label{eq:r-final}
\end{equation}



% vim: set ff=unix tw=79 sw=4 ts=4 et ic ai :
