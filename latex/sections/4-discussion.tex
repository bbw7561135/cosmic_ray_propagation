\section{Discussion}
\subsection{The Monte-Carlo method}
The Monte-Carlo approach can be considered as rather powerful, since it allows
to perform  an experiment -- which might be difficult or even impossible in a
laboratory -- a large amount of times independently of prior results. Because of
the central limit theorem -- which states, that the distribution of a
large amount of randomly generated, independent experiments tends
towards a normal distribution (as seen in \cref{sec:eval-circle}) -- one
can gain significant insight into the system under consideration,
even with its interactions being subject to random fluctuations.
This is particularly helpful for complex systems, which cannot completely be
modeled analytically, but are instead accessible through a probabilistic
description.

\subsection{Reweighting}
A simple and effective technique to obtain physically appropriate and
consistently accurate results, by not wasting CPU-time due to unevenly sampling
the available energies.

\subsection{Nuclear Interactions}


\subsection{Magnetic Deflection}


\subsection{Constraints on the Sources}


% vim: set ff=unix tw=79 sw=4 ts=4 et ic ai :
