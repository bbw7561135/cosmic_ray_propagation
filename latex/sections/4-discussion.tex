\section{Discussion}
\subsection{The Monte-Carlo method}
The Monte-Carlo approach can be considered as rather powerful, since it allows
to perform  an experiment -- which might be difficult or even impossible in a
laboratory -- a large amount of times independently of prior results. Because of
the central limit theorem -- which states, that the distribution of a
large amount of randomly generated, independent experiments tends
towards a normal distribution (as seen in \cref{sec:eval-circle}) -- one
can gain significant insight into the system under consideration,
even with its interactions being subject to random fluctuations.
This is particularly helpful for complex systems, which cannot completely be
modeled analytically, but are instead accessible through a probabilistic
description.

\subsection{Reweighting}
A simple and effective technique to obtain physically appropriate and
consistently accurate results, by not wasting CPU-time due to unevenly sampling
the available energies.

\subsection{Nuclear Interactions}
The estimated changes in the spectral indices due to the different interaction
mechanisms are qualitatively as expected from theory and show that an initial
Fermi spectrum can be changed into something closer to the observations.
However the thresholds at which certain features emerge are throughout about
one magnitude in energy too large.  With the steering scripts appearing
correct, the suspicion of an off-by-one error in the \CRPropa~source is
aroused.

The introduced loss time scale proves useful for comparisons of different
simulations within this work and allows to investigate certain features in the
obtained data more closely, yet it lacks physical meaning (giving raise to
velocities $>c$; $\d_{\mathrm{scale}}$ being chosen so that the numbers
\enquote{look nicely and comparably}).

Speaking about the differences between the two candidate types, most
observations seem to agree with explanations of the IRB having a significantly
lower photon density and iron having a significantly higher cross section.
Especially iron being treated as a bunch of single protons and neutrons fits in
the picture.

Remarking conclusively, it is rather astonishing how much information can be
drawn from such a \enquote{simple} one-dimensional simulation and how much can
therewith be learned about the physics of the situation.


\subsection{Magnetic Deflection}


\subsection{Constraints on the Sources}


% vim: set ff=unix tw=79 sw=4 ts=4 et ic ai :
