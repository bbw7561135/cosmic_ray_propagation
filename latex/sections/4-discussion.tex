\section{Discussion}
\subsection{The Monte-Carlo method}
The Monte-Carlo approach can be considered as rather powerful, since it allows
to perform  an experiment -- which might be difficult or even impossible in a
laboratory -- a large amount of times independently of prior results. Because of
the central limit theorem -- which states, that the distribution of a
large amount of randomly generated, independent experiments tends
towards a normal distribution (as seen in \cref{sec:eval-circle}) -- one
can gain significant insight into the system under consideration,
even with its interactions being subject to random fluctuations.
This is particularly helpful for complex systems, which cannot completely be
modeled analytically, but are instead accessible through a probabilistic
description.

\subsection{Reweighting}
\label{sec:dis-rew}
A simple and effective technique to obtain physically appropriate and
consistently accurate results, by not wasting CPU time due to unevenly sampling
the available energies.

\subsection{Nuclear Interactions}
The estimated changes in the spectral indices due to the different interaction
mechanisms are qualitatively as expected from theory and show that an initial
Fermi spectrum can be changed into something closer to the observations.
However the thresholds at which certain features emerge are throughout about
one magnitude in energy too large.  With the steering scripts appearing
correct, the suspicion of an off-by-one error in the \CRPropa~source is
aroused.

The introduced loss time scale proves useful for comparisons of different
simulations within this work and allows to investigate certain features in the
obtained data more closely, yet it lacks physical meaning (giving raise to
velocities $>c$; $\d_{\mathrm{scale}}$ being chosen so that the numbers
\enquote{look nice and comparable}).

Speaking about the differences between the two candidate types, most
observations seem to agree with explanations of the IRB having a lower photon
density and iron having a higher cross section. Especially iron being treated
as a bunch of single protons and neutrons fits in the picture.

Remarking conclusively, it is rather astonishing how much information can be
drawn from such a \enquote{simple} one-dimensional simulation and how much can
therewith be learned about the physics of the situation.


\subsection{Magnetic Deflection}
At first it is to note that the results are partly non-physical (when related
to the current age of the universe) as \dmax~was chosen slightly too large,
however it is still in the same order of magnitude as \dAoU~and does not
change the significance of the results (it should suffice to simply disregard
the last bin of the trajectory length distribution when conducting further
analysis). For future simulations it is advisable to avoid such an error with
an eye on CPU time.

When investigating the resonant interval, one finds that its upper limit first
coincides approximately with the expected range $0.075<\rho<1$, but is shifted
to higher rigidities with increasing radii (for
$\Robserver=\SI{48}{\mega\parsec}$ the peak starts at $\rho\sim\num{50}$). In
order to obtain the full peak and be able to have a look at the lower limit,
one would have to re-run the simulations with lower \Emin.
As to why this shift appears, no clear explanation seems obvious other than
uncertainties due to approximations in the theoretical derivation.

The results show that a turbulent magnetic field in the intergalactic medium is
able by resonant scattering to transform an initial Fermi spectrum into
something more similar to the observations and to produce an isotropic
distribution at the observer. This is the case for
$\Robserver\ge\SI{2.4}{\mega\parsec}$ with $\Brms\ge\SI{10}{\nano\gauss}$ and
(as an interesting possibility including anisotropies at higher energies)
$\Robserver\ge\SI{24}{\mega\parsec}$ with $\Brms=\SI{1}{\nano\gauss}$.

Some additional questions, whose investigation might be interesting in this
context:
\begin{itemize}
    \item Which value for $\Brms$ can be regarded as realistic? Is there a way
        to theoretically estimate this quantity, perhaps in conjunction with
        the plasma dynamics of the IGM?

    \item A more thorough statistical investigation of the trajectory length
        and angular distribution would be interesting. Can an underlying
        probability density function be identified and a deeper connection to
        phenomena such as random walks be found?

    \item As the obtained distributions behave rather similar (especially for
        sufficiently large \Robserver~and $\Brms\ge\SI{10}{\nano\gauss}$), the
        question arises whether it would be adequate to simulate one case
        (fixed \Brms) and obtain the other cases through the application of an
        appropriate weight function. This would yield the possibility to
        investigate a broader range of parameters, while saving CPU time (see
        \cref{sec:dis-rew}).

\end{itemize}


\subsection{Constraints on the Sources}


% vim: set ff=unix tw=79 sw=4 ts=4 et ic ai :
