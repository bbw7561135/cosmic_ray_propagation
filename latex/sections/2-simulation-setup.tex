\section{Simulation setup}
\subsection{The Monte-Carlo method}
In order to demonstrate the abilities of the Monte-Carlo approach, the area
of the unit circle is to be computed stochastically. This is done by
randomly sampling \num{10000} coordinates $(x,y)$ from $\mathcal{U}(0,1)$
-- the continuous uniform distribution over the interval $[0,1]$ -- and
counting those, which have a euclidean distance $<1$ from the origin (see
\cref{fig:circle-area-scatter}).

\begin{figure}[ht]
    \centering
    \includegraphics[width=.6\textwidth]{circle-area}
    \caption{Scatter plot of uniformly distributed random points. The circle
    area obtained from this sample is $A=3.128$.}
    \label{fig:circle-area-scatter}
\end{figure}

Because the points are distributed uniformly, the ratio of points in the
square (\ie~all of them) and points in the quadrant of the circle equals the
ratio of the square's and the quadrant's surface area:
\begin{equation*}
    \frac{N_{\mathrm{quadrant}}}{N_{\mathrm{total}}}
    =\frac{A_{\mathrm{quadrant}}}{A_{\mathrm{square}}}
\end{equation*}
With $A_{\mathrm{square}}$ known to be one, the circle's surface area can
be calculated:
\begin{equation}
    A_{\mathrm{circle}}=4\frac{N_{\mathrm{quadrant}}}{N_{\mathrm{total}}}
    \label{eq:circle-area-mc}
\end{equation}

This simulation is then repeated \num{1000} times in order to
obtain a dataset whose distribution and accuracy can be investigated.
This can easily be done with \texttt{numpy}:
\begin{minted}[%
    frame=lines,
    framesep=2mm,
    baselinestretch=1.2,
    bgcolor=black!20!white,
    fontsize=\small,
]{python}
    >>> r = np.random.uniform(size=(2, NUMBER_OF_SAMPLES, NUMBER_OF_MC_RUNS))
    >>> area = 4 * np.count_nonzero(r[0]**2 + r[1]**2 < 1, axis=0) / NUMBER_OF_SAMPLES
    >>> area.mean(), area.std()
\end{minted}


\subsection{Reweighting}
In order to get familiar with \CRPropa~and to demonstrate the reweighting, a
simple simulation with the following setup is performed:

Protons are propagated in 1D while neglecting interaction mechanisms and
magnetic fields (the resulting energy spectrum at the observer will remain
unchanged from the initial spectrum; in particular all propagated particles
are detected bt the observer). Their sources are uniformly distributed
between redshifts of $z=0$ and $z=2$ and emit particles according to a
power-law energy spectrum from the interval \SIrange{1e18}{1e21}{\eV}. The
simulation is performed once with a spectral index $\alpha=-1$ responsible for
a flat energy distribution and once with a spectral index $\alpha=-2$ as
expected from first-order Fermi acceleration, each time using \num{1000}
particles.

\emph{Note:} All following simulations will be performed with a spectral index
$\alpha=-1$ and their results will be reweighted as needed.


\subsection{Nuclear Interactions}
\todo{remove photo-disint for protons}
There are several extragalactic energy loss processes implemented in
\CRPropa, whose characteristics are to be investigated for H and \Fe~nuclei on
close (\SIrange{0}{10}{\mega\parsec}) and far
(\SIrange{100}{1000}{\mega\parsec}) distances. For each of the four settings,
eight simulations are performed, separately considering the individual
processes:
\begin{itemize}
    \item without interaction
    \item Redshift
    \item Photo-Pion-Production
    \item Photo-Disintegration
    \item Electron-Pair-Production
\end{itemize}
where the last three processes are due to interactions with a given photon
field (CMB, IRB). With \Fe, nuclear decay is considered additionally in all
cases.

The domain is 1D with uniformly distributed sources in the given distance
interval.
The emitted particles are drawn from a flat power-law spectrum from the
interval \SI{1e17}{\eV} to $\text{Z}\times10^{21}\si{\eV}$ (considering the
total charge of the given nuclei, therewith causing iron to have a higher
maximal energy compared to protons), where those which fall below the minimum
energy \SI{1e17}{\eV} are discarded. Each time \num{10000} candidates are
propagated; all particles and their secondaries (in case of \Fe) are detected.


\subsection{Magnetic Deflections}
\label{sec:setup-defl}
In order to investigate deflections of CR protons by a turbulent
magnetic field, a simulation with a 3D domain is set up. \CRPropa~ships the
necessary tools to create a 3D periodic grid (covering arbitrary volumes) on
which a turbulent magnetic field according to Kolmogorov theory can be
generated.

The grid size is $256^3$ with a \SI{75}{\kilo\parsec} grid spacing, resulting
in an extend of $(256\times\SI{75}{\kilo\parsec})^3$. The turbulence of the
magnetic field behaves according to a power spectrum with spectral index
$-11/3$ (instead of $-5/3$ as mentioned in \cref{sec:intro-turbulence},
according to the manual) with a minimum and maximum length scale of
\SI{170}{\kilo\parsec} and \SI{2000}{\kilo\parsec} respectively. Thus the
resulting correlation length is \todox{value}$l_c=$.
There are four simulations performed with varying root mean square field
strengths $\Brms=\SIlist{50;20;10;1}{\nano\gauss}$.

The source is now point-like and located at the center of the grid. It
isotropically ejects protons with energies from a flat power-law spectrum,
ranging from \SIrange{1e17}{1e20}{\electronvolt}. Instead of a single
point-like observer, six observer spheres arranged concentrically around the
source with radii
$r_{\mathrm{sphere}}=[\numlist[list-final-separator={,}]{.05;.5;1;5;50;100}]\times
l_c$ are defined, which detect \todox{all fields?}coordinates, momentum and
energy of leaving particles. The particles are not removed on detection and
thus can re-enter the spheres (however, due to implementation details, only
leaving particles are detected). Further a maximum trajectory length
$d_{\mathrm{max}}=\num{5e3}\times l_c$ (\ie~two orders of magnitude larger than
the second largest observer sphere) is set to provide a termination condition
for the simulation.

All additional nuclear interaction mechanisms are turned of, in order to solely
record the effects of the magnetic field.
For each of the four runs, \num{10000} candidates are propagated.


\subsection{Constraints on the Sources}
The final task is to put parametrical constrains on possible CR sources. To
obtain the data, the setup described in \cref{sec:setup-defl} is extended by
adding nuclear interaction mechanisms for protons (Redshift,
Photo-Pion-Production and Photo-Pair-Production; the last two once in
interaction with the CMB and once with the IRB), while leaving all other
parameters unchanged.

Again four simulations for varying \Brms~and \num{10000} particles each are
performed.


% vim: set ff=unix tw=79 sw=4 ts=4 et ic ai :
